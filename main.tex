% Arquivo LaTeX de exemplo de dissertação/tese a ser apresentados à CPG do IME-USP
% 
% Versão 5: Sex Mar  9 18:05:40 BRT 2012
%
% Criação: Jesús P. Mena-Chalco
% Revisão: Fabio Kon e Paulo Feofiloff
%  
% Obs: Leia previamente o texto do arquivo README.txt

\documentclass[12pt,a4paper]{book}

% ---------------------------------------------------------------------------- %
% Pacotes

\usepackage[T1]{fontenc}
\usepackage[brazil]{babel}
\usepackage{nomencl}
\usepackage[fixlanguage]{babelbib}
\usepackage[utf8]{inputenc}
\usepackage[pdftex]{graphicx}           % usamos arquivos pdf/png como figuras
\usepackage{setspace}                   % espaçamento flexível
\usepackage{indentfirst}                % indentação do primeiro parágrafo
\usepackage{subcaption}

\usepackage{makeidx}                    % índice remissivo
\usepackage[nottoc]{tocbibind}          % acrescentamos a bibliografia/indice/conteudo no Table of Contents
\usepackage{courier}                    % usa o Adobe Courier no lugar de Computer Modern Typewriter
\usepackage{type1cm}                    % fontes realmente escaláveis
\usepackage{listings}                   % para formatar código-fonte (ex. em Java)
\usepackage{titletoc}
%\usepackage[bf,small,compact]{titlesec} % cabeçalhos dos títulos: menores e compactos

\usepackage[width=0.6\textwidth ,font={small, sf,it},labelfont={color=blue,bf},
labelsep=endash, format=plain,labelsep=period]{caption}
\usepackage[dvipsnames*,svgnames,x11names]{xcolor}
\usepackage[a4paper,top=2.54cm,bottom=2.0cm,left=2.0cm,right=2.54cm]{geometry} % margens
%\usepackage[pdftex,plainpages=false,pdfpagelabels,pagebackref,colorlinks=true,citecolor=black,linkcolor=black,urlcolor=black,filecolor=black,bookmarksopen=true]{hyperref} % links em preto
\usepackage[pdftex,plainpages=false,pdfpagelabels,pagebackref,colorlinks=true,citecolor=DarkGreen,linkcolor=NavyBlue,urlcolor=DarkRed,filecolor=green,bookmarksopen=true]{hyperref} % links coloridos
\usepackage[all]{hypcap}                % soluciona o problema com o hyperref e capitulos
\usepackage[square,sort,nonamebreak,comma,numbers]{natbib}  % citação bibliográfica alpha (alpha-ime.bst)
\fontsize{60}{62}\usefont{OT1}{cmr}{m}{n}{\selectfont}



\usepackage{amsthm}
\usepackage{srcltx}
\usepackage{multicol}
\usepackage{array}
\usepackage{enumerate}   
\usepackage{amssymb}
\usepackage{mathtools} %%symbols as :=
\usepackage{tikz} %% permite tikz
\usepackage{svg}    %%ibagens
\usepackage{ragged2e} 	

% ---------------------------------------------------------------------------- %
% Cabeçalhos similares ao TAOCP de Donald E. Knuth
\usepackage{fancyhdr}
\pagestyle{fancy}
\fancyhf{}
\renewcommand{\chaptermark}[1]{\markboth{\MakeUppercase{#1}}{}}
\renewcommand{\sectionmark}[1]{\markright{\MakeUppercase{#1}}{}}
\renewcommand{\headrulewidth}{0pt}

% ---------------------------------------------------------------------------- %
\graphicspath{{./figuras/}}             % caminho das figuras (recomendável)
\frenchspacing                          % arruma o espaço: id est (i.e.) e exempli gratia (e.g.) 
\urlstyle{same}                         % URL com o mesmo estilo do texto e não mono-spaced
\makeindex                              % para o índice remissivo
\raggedbottom                           % para não permitir espaços extra no texto
\fontsize{60}{62}\usefont{OT1}{cmr}{m}{n}{\selectfont}
\cleardoublepage
\normalsize


\newcommand{\K}{K_{\ell,r}} %%add by Paulo
\newcommand{\Slr}{\mathcal{S}_{\ell,r}} %%% add by Paulo


\let\phi=\varphi


\def\til{\tilde}


\def\NN{\mathds N}
\def\ZZ{\mathds Z} 
\def\QQ{\mathds Q}
\def\RR{\mathds R}
\def\PP{\mathds P}
\def\EE{\mathds E}

\def\cB{\mathcal B}
\def\cR{\mathcal R}

\def\ra{\longrightarrow}
\def\red{\text{\rm red}}
\def\blue{\text{\rm blue}}
\def\R{\text{\rm red}}
\def\B{\text{\rm blue}}

\def\T{{\mathcal T}}



\def\mcarrow{\xrightarrow[]{\raisebox{0.0mm}[0mm]{$\scriptstyle \rm mr$}}}

\def\rbarrow{\xrightarrow[]{\raisebox{0.0mm}[0mm]{$\scriptstyle \rm rb$}}}
\newcommand{\rb}{\xrightarrow{\text{\rm rb}}}


\let\eps=\varepsilon
\let\theta=\vartheta
\let\rho=\varrho
\let\phi=\varphi


\let\epsilon\varepsilon
\let\:\colon
\let\subset\subseteq
\let\supset\supseteq

\let\<\langle
\let\>\rangle
\def\llfloor{\left\lfloor}
\def\rrfloor{\right\rfloor}
\def\llceil{\left\lceil}
\def\rrceil{\right\rceil}
\let\ot\leftarrow
\def\set{^{\{\}}}

\def\pmc#1{p^{\rm rb}_{#1}}



\newtheorem{teorema}{Teorema}[chapter]       
\newtheorem{Afirmativa}[teorema] {Afirmativa}         
\newtheorem{lema}      [teorema] {Lema}         
\newtheorem{corolario} [teorema] {Corolário}     
\newtheorem{proposicao} [teorema] {Proposição}     
\newtheorem{fato}      [teorema] {Fato}          
\newtheorem{conjectura}[teorema] {Conjectura}    
\newtheorem{problema}  [teorema] {Problema}       
\newtheorem{questao}[teorema] {Questão}    
\newtheorem{aff}[teorema]{Afirmação}
\newtheorem{obs}[teorema]{Observação}
\newtheorem{propriedade}  [teorema] {Propriedade}       
\newtheorem{defi}  [teorema] {Definição}       
\newtheorem{exercicio}  [teorema] {Exercício}

\def\bba{{\mathbb A}}
\def\bbb{{\mathbb B}}
\def\bbc{{\mathbb C}}
\def\bbd{{\mathbb D}}
\def\bbe{{\mathbb E}}
\def\bbf{{\mathbb F}}
\def\bbg{{\mathbb G}}
\def\bbh{{\mathbb H}}
\def\bbi{{\mathbb I}}
\def\bbj{{\mathbb J}}
\def\bbk{{\mathbb K}}
\def\bbl{{\mathbb L}}
\def\bbm{{\mathbb M}}
\def\bbn{{\mathbb N}}
\def\bbo{{\mathbb O}}
\def\bbp{{\mathbb P}}
\def\bbq{{\mathbb Q}}
\def\bbr{{\mathbb R}}
\def\bbs{{\mathbb S}}
\def\bbt{{\mathbb T}}
\def\bbu{{\mathbb U}}
\def\bbv{{\mathbb V}}
\def\bbw{{\mathbb W}}
\def\bbx{{\mathbb X}}
\def\bby{{\mathbb Y}}
\def\bbz{{\mathbb Z}}

\def\cala{{\mathcal A}}
\def\calb{{\mathcal B}}
\def\calc{{\mathcal C}}
\def\cald{{\mathcal D}}
\def\cale{{\mathcal E}}
\def\calf{{\mathcal F}}
\def\calg{{\mathcal G}}
\def\hh{{\mathcal H}}
\def\cali{{\mathcal I}}
\def\calj{{\mathcal J}}
\def\calk{{\mathcal K}}
\def\call{{\mathcal L}}
\def\calm{{\mathcal M}}
\def\caln{{\mathcal N}}
\def\calo{{\mathcal O}}
\def\calp{{\mathcal P}}
\def\calq{{\mathcal Q}}
\def\calr{{\mathcal R}}
\def\cals{{\mathcal S}}
\def\calt{{\mathcal T}}
\def\calu{{\mathcal U}}
\def\calv{{\mathcal V}}
\def\calw{{\mathcal W}}
\def\calx{{\mathcal X}}
\def\caly{{\mathcal Y}}
\def\calz{{\mathcal Z}}





\def\ex{\mathop{\text{\rm ex}}\nolimits}
\def\exdeg{\mathop{\text{\rm exdeg}}\nolimits}
\def\circum{\mathop{\text{\rm circ}}\nolimits}
\def\PA{\mathop{\text{\rm PA}}\nolimits}
\def\posto{\mathop{\text{\rm posto}}\nolimits}
\def\ci{\mathop{\text{\rm ci}}\nolimits}
\def\BDD{\mathop{\text{\rm LMT}}\nolimits}
\def\TUPLE{\mathop{\text{\rm CG}}\nolimits}
\def\col{\mathop{\text{\rm col}}\nolimits}
\def\AR{\mathop{\text{\rm AR}}\nolimits}


\newcommand{\sr}{\hat{r}}
\def\bu{\mathop{\text{\rm bu}}\nolimits}

\newcommand{\orr}{\overrightarrow{r}} %%oriented size-ramsey symbols
\newcommand{\orq}{\overrightarrow{r_q}}


\makeatletter
\def\@setaddresses{\par
  \nobreak \begingroup
\footnotesize
  \def\author##1{\nobreak\addvspace\bigskipamount}%
  \def\\{\unskip, \ignorespaces}%
  \interlinepenalty\@M
  \def\address##1##2{\begingroup
    \par\addvspace\bigskipamount\indent
    \@ifnotempty{##1}{(\ignorespaces##1\unskip) }%
    {\scshape\ignorespaces##2}\par\endgroup}%
  \def\curraddr##1##2{\begingroup
    \@ifnotempty{##2}{\nobreak\indent{\itshape Current address}%
      \@ifnotempty{##1}{, \ignorespaces##1\unskip}\/:\ccpace
      ##2\par}\endgroup}%
  \def\email##1##2{\begingroup
    \@ifnotempty{##2}{\nobreak\indent{\itshape Endereços Eletrônicos}%
      \@ifnotempty{##1}{, \ignorespaces##1\unskip}\/:\ccpace
      \ttfamily##2\par}\endgroup}%
  \def\urladdr##1##2{\begingroup
    \@ifnotempty{##2}{\nobreak\indent{\itshape URL}%
      \@ifnotempty{##1}{, \ignorespaces##1\unskip}\/:\ccpace
      \ttfamily##2\par}\endgroup}%
  \addresses
  \endgroup
}
\makeatother

\def\({\left(}
\def\){\right)}  
\def\<{\langle}
\def\>{\rangle}
\let\:=\colon

\def\PP{{\mathbb P}}
\def\EE{{\mathbb E}}
\def\ZZ{{\mathbb Z}}
\def\NN{{\mathbb N}}
\def\QQ{{\mathbb Q}}

\def\cP{{\mathcal P}}
\def\cL{{\mathcal L}}
\def\bfa{\textbf{a}}
\def\bfb{\textbf{b}}
\def\bfx{\textbf{x}}
\def\bfzero{\textbf{0}}

\def\ex{\mathop{\text{\rm ex}}\nolimits}
\def\exdeg{\mathop{\text{\rm exdeg}}\nolimits}
\def\circum{\mathop{\text{\rm circ}}\nolimits}
\def\PA{\mathop{\text{\rm PA}}\nolimits}
\def\posto{\mathop{\text{\rm posto}}\nolimits}
\def\ci{\mathop{\text{\rm ci}}\nolimits}
\def\BDD{\mathop{\text{\rm LMT}}\nolimits}
\def\TUPLE{\mathop{\text{\rm CG}}\nolimits}
\def\col{\mathop{\text{\rm col}}\nolimits}











% ---------------------------------------------------------------------------- %
% Opções de listing usados para o código fonte
% Ref: http://en.wikibooks.org/wiki/LaTeX/Packages/Listings
\lstset{ %
language=Java,                  % choose the language of the code
basicstyle=\footnotesize,       % the size of the fonts that are used for the code
numbers=left,                   % where to put the line-numbers
numberstyle=\footnotesize,      % the size of the fonts that are used for the line-numbers
stepnumber=1,                   % the step between two line-numbers. If it's 1 each line will be numbered
numbersep=5pt,                  % how far the line-numbers are from the code
showspaces=false,               % show spaces adding particular underscores
showstringspaces=false,         % underline spaces within strings
showtabs=false,                 % show tabs within strings adding particular underscores
frame=single,	                % adds a frame around the code
framerule=0.6pt,
tabsize=2,	                    % sets default tabsize to 2 spaces
captionpos=b,                   % sets the caption-position to bottom
breaklines=true,                % sets automatic line breaking
breakatwhitespace=false,        % sets if automatic breaks should only happen at whitespace
escapeinside={\%*}{*)},         % if you want to add a comment within your code
backgroundcolor=\color[rgb]{1.0,1.0,1.0}, % choose the background color.
rulecolor=\color[rgb]{0.8,0.8,0.8},
extendedchars=true,
xleftmargin=15pt,
xrightmargin=15pt,
framexleftmargin=15pt,
framexrightmargin=15pt
}

% ---------------------------------------------------------------------------- %
% Corpo do texto
\begin{document} 
\frontmatter 
% cabeçalho para as páginas das seções anteriores ao capítulo 1 (frontmatter)
\fancyhead[RO]{{\footnotesize\rightmark}\hspace{2em}\thepage}
\setcounter{tocdepth}{2}
\setcounter{secnumdepth}{3}
\fancyhead[LE]{\thepage\hspace{2em}\footnotesize{\leftmark}}
\fancyhead[RE,LO]{}
\fancyhead[RO]{{\footnotesize\rightmark}\hspace{2em}\thepage}

\doublespacing %\onehalfspacing  % espaçamento

% ---------------------------------------------------------------------------- %
% CAPA
% Nota: O título para as dissertações/teses do IME-USP devem caber em um 
% orifício de 10,7cm de largura x 6,0cm de altura que há na capa fornecida pela SPG.
\thispagestyle{empty}
\begin{center}
    \vspace*{2cm}
    \textbf{\Large{Padrões de coloração de arestas e propriedades anti-Ramsey }}\\
    
    \vspace*{1.1cm}
    \Large{Paulo Matias da Silva Junior}    
    \vskip 2cm
    \textsc{
    Tese apresentada\\[-0.3cm] 
    ao\\[-0.3cm]
    Centro de Matemática, Cognição e Computação\\[-0.3cm]
    da\\[-0.3cm]
    Universidade Federal do ABC\\[-0.3cm]
    para\\[-0.3cm]
    obtenção do título\\[-0.3cm]
    de\\[-0.3cm]
    Doutor em Ciência da Computação}
    
    \vskip 1.3cm
    Programa: Ciência da Computação\\
    Orientador: Prof. Dr. Guilherme Oliveira Mota
   	\vskip 1cm
    \normalsize{Durante o desenvolvimento deste trabalho o autor recebeu auxílio
    financeiro da FAPESP}    
    \vskip 0.4cm
    \normalsize{Santo André, Janeiro de 2024}
\end{center}


% ---------------------------------------------------------------------------- %
% Página de rosto (SÓ PARA A VERSÃO DEPOSITADA - ANTES DA DEFESA)
% Resolução CoPGr 5890 (20/12/2010)
%
% IMPORTANTE:
%   Coloque um '%' em todas as linhas
%   desta página antes de compilar a versão
%   final, corrigida, do trabalho
%
%
\newpage
\thispagestyle{empty}
    \begin{center}
        \vspace*{2.3 cm}
        \textbf{\Large{Padrões de coloração de arestas e propriedades anti-Ramsey }}\\
        \vspace*{2 cm}
    \end{center}

    \vskip 2cm

    \begin{flushright}
	Esta é a versão original da tese elaborada pelo\\
	candidato Paulo Matias da Silva Junior, tal como \\
	submetida à Comissão Julgadora.
    \end{flushright}

\pagebreak

% ---------------------------------------------------------------------------- %
% Página de rosto (SÓ PARA A VERSÃO CORRIGIDA - APÓS DEFESA)
% Resolução CoPGr 5890 (20/12/2010)
%
% Nota: O título para as dissertações/teses do IME-USP devem caber em um 
% orifício de 10,7cm de largura x 6,0cm de altura que há na capa fornecida pela SPG.
%
% IMPORTANTE:
%   Coloque um '%' em todas as linhas desta
%   página antes de compilar a versão do trabalho que será entregue
%   à Comissão Julgadora antes da defesa
%
%
\newpage
\thispagestyle{empty}
    \begin{center}
        \vspace*{2.3 cm}
        \textbf{\Large{Padrões de coloração de arestas e propriedades anti-Ramsey }}\\
        \vspace*{2 cm}
    \end{center}

    \vskip 2cm

    \begin{flushright}
	Esta versão da dissertação/tese contém as correções e alterações sugeridas\\
	pela Comissão Julgadora durante a defesa da versão original do trabalho,\\
	realizada em dd/01/2024.
    %Uma cópia da versão original está disponível no\\ Instituto de Matemática e Estatística da Universidade de São Paulo.

    \vskip 2cm

    \end{flushright}
    \vskip 4.2cm

    \begin{quote}
    \noindent Comissão Julgadora:
    
    \begin{itemize}
		\item Prof. Dr. Guilherme Oliveira Mota (orientador) - IME-USP 
		\item Profª. Drª. Nome Completo - IME-USP %[sem ponto final]
		\item Prof. Dr. Nome Completo - IMPA %[sem ponto final]
         \item Prof. Dr. Nome Completo - UFBA %[sem ponto final]
        \item Profª. Drª. Nome Completo - UFABC %[sem ponto final]
    \end{itemize}
      
    \end{quote}
\pagebreak

\pagenumbering{roman}     % começamos a numerar 

% ---------------------------------------------------------------------------- %
% Agradecimentos:
% Se o candidato não quer fazer agradecimentos, deve simplesmente eliminar esta página 
 \chapter*{Agradecimentos}

Agradeço a Deus pela vida, a vida conjuga a matemática e vice-versa.

Tenho gratidão sem medidas por minha mãe, a querida Dona Geisa, que batalhou pelos cinco filhos, muito tempo sozinha, e ainda os apoia até mesmo com eles crescidos. 
Agradeço a toda minha família que, agitada como é, sempre me surpreende.

Agradeço ao professor Guilherme Mota pela orientação, por ser sempre gentil e acreditar no meu potencial, além da visão ampla de ciência colaborativa que ele me apresentou.

Preciso agradecer àqueles que caminharam comigo, destaco: Lucas Colucci pelas conversas sempre matemáticas; Giovani pelo papo cheio de histórias e Ariel pelos debates de assuntos complicados da vida.

Por fim, agradeço à Fundação de Amparo à Pesquisa do Estado de São Paulo pela concessão de bolsas de estudo para a realização deste trabalho.

% ---------------------------------------------------------------------------- %
% Resumo
\chapter*{Resumo}

 Este é o texto tem como foco o estudo de propriedades anti-Ramsey de grafos aleatórios e determinísticos.
 Primeiro,
consideramos o problema proposto por Conlon e Tyomkyn, de determinar o menor número de cores necessárias para se colorir propriamente as arestas do grafo completo $K_n$,
de modo que nenhum par de triângulos vértices-disjuntos de $K_n$ tenha a mesma tripla de cores. 
Esta quantidade de cores era desconhecida para $n$ par e conseguimos computá-la para alguns casos pequenos e também para uma família infinita de valores de $n$.

Consideramos também o seguinte problema do tipo anti-Ramsey: dados grafos $G$ e $H$, denotamos por $G \rbarrow H$ a propriedade em que toda coloração própria de $E(G)$ apresenta uma cópia rainbow de $H$ em $G$.
Provamos que o valor da função limiar para $G(n,p) \rbarrow K_{\ell,r}$ é dado por $n^{(\ell+r-2)/(\ell r-1)}$, para grafos bipartidos completos $\K$ com $\ell,r \geq 3$. Provamos também que o limiar é assintoticamente menor que este valor se $\ell=2$ e $r\geq 2$.
\\

\noindent \textbf{Palavras-chave:} anti-Ramsey, coloração própria, função limiar, grafos bipartidos.

% ---------------------------------------------------------------------------- %
% Abstract
\chapter*{Abstract}
%% \noindent MATIAS, P. \textbf{Mysterious title in English}. 2010. 120 f.  Tese (Doutorado) - Centro de Matemática, Cognição e Computação,Universidade Federal do ABC, Santo André, 2023.\\

% Elemento obrigatório, elaborado com as mesmas características do resumo em língua portuguesa. De acordo com o Regimento da Pós- Graduação da USP (Artigo 99), deve ser redigido em inglês para fins de divulgação. 
This text focuses on the study of anti-Ramsey properties of random and deterministic graphs.
  First, we consider the problem proposed by Conlon and Tyomkyn, of determining the smallest number of colors necessary to obtain a proper edge coloring of the complete graph $K_n$, so that no pair of vertex-disjoint triangles of $K_n$ have the same triple of colors.
This number of colors was unknown for $n$ even and we were able to compute it for some small cases and also for an infinite family of $n$ values.

We also consider the following anti-Ramsey problem: given graphs $G$ and $H$, we denote by $G \rbarrow H$ the property in which every proper coloring of $E(G)$ contains a rainbow copy of $H $ in $G$.
We prove that the value of the threshold function for $G(n,p) \rbarrow K_{\ell,r}$ is given by $n^{(\ell+r-2)/(\ell r-1)}$ , for complete bipartite graphs $\K$ with $\ell,r \geq 3$. We also prove that the threshold is asymptotically smaller than this value if $\ell=2$ and $r\geq 2$.
\\

\noindent \textbf{Keywords:} anti-Ramsey, proper edge coloring, threshold function, bipartite graphs.

% ---------------------------------------------------------------------------- %
% Sumário
\tableofcontents    % imprime o sumário

% ---------------------------------------------------------------------------- %
% Listas de figuras e tabelas criadas automaticamente
\listoffigures            
% \listoftables            

% ---------------------------------------------------------------------------- %
% Capítulos do trabalho
\mainmatter

% cabeçalho para as páginas de todos os capítulos
\fancyhead[RE,LO]{\thesection}

%singlespacing              % espaçamento simples
\onehalfspacing            % espaçamento um e meio




\chapter{Introdução}
\label{sec:intro}

%Este trabalho de doutoramento, elaborado pelo aluno de doutorado Paulo Matias da Silva Junior, no Centro de Matemática, Computação e Cognição, UFABC, enquadra-se na área de combinatória extremal.

A Teoria de Ramsey é uma área da matemática que investiga a existência de ordem de subconjuntos dentro de conjuntos grandes de objetos. 
O nome desse campo se deve ao resultado clássico de Frank P. Ramsey \cite{Ra} que, em 1930, enquanto estudava um problema de lógica proposicional, ele provou o conhecido ``Teorema de Ramsey'' no qual afirma que para quaisquer inteiros positivos $t$ e $q$, existe um inteiro $n$ tal que toda $q$-coloração de arestas do grafo completo $K_n$ contém uma cópia monocromática de $K_t$.

% Dentre as técnicas que pretendemos utilizar para atacar os problemas propostos, destacamos o Lema de Regularidade de Szemerédi e suas variantes (veja, por exemplo, \cite{gerke05:_sparse_survey, Kohayakawa97Szemeredi, KomlosSimonovits96, roedl05:PNAS, szemeredi76,  roedl10+:_graph_regul_lemmas, roedl07:_count,roedl07:_regul}). 

Dados grafos $G$ e $H$ e um inteiro positivo $q$, dizemos que $G$
\emph{é $q$-Ramsey para} $H$ se toda $q$-coloração das arestas de $G$ contém uma cópia monocromática de $H$.
Denotamos essa propriedade por $G\rightarrow (H)_q$.
Quando $q=2$, escrevemos simplesmente $G\rightarrow H$.
%Em sua forma mais simples, o clássico Teorema de Ramsey \cite{Ra} afirma que para qualquer grafo $H$ existe um inteiro $N$ tal que $K_N \rightarrow H$.
O \emph{número de Ramsey}~$r(H)$ de um grafo~$H$ é definido como o menor inteiro~$N$  tal que $K_N \rightarrow H$.

Uma generalização natural do número de Ramsey com mais cores na coloração do grafo completo é buscar cópias monocromáticas de múltiplos grafos.
Com isso, chama-se de \emph{número de Ramsey de ordem} $q$, denotado por $r_q(H_1,H_2, \ldots ,H_q)$,
 o menor inteiro positivo $N$ tal que, se as arestas de um grafo completo $K_N$ são particionadas em $q$ classes de cores distintas, fornecendo $q$ subgrafos $G_1, G_2, \ldots,G_q$, então,
pelo menos um dos grafos $G_i$ contém um subgrafo isomorfo a $H_i$, onde $1 \leq i \leq q$.
Problemas em Teoria de Ramsey têm sido muito estudados e técnicas elegantes têm sido desenvolvidas para estimar os números de Ramsey, especialmente de ordens dois \cite{pontiveros2016ramsey, montgomery2023ramsey} e três \cite{BeSk09, MoSaScTa13}, formando uma área ativa de pesquisa.
O trabalho de Conlon, Fox e Sudakov~\cite{CoFoSu15} contém um resumo dos avanços na área de Ramsey, enquanto Radziszowski \cite{radziszowski2011small} possui um artigo resumo com constante atualização para o número de Ramsey de grafos pequenos.

Uma variante de problema tipo Ramsey proposta por Erd\H{o}s, Faudree, Rousseau e Schelp~\cite{ErFaRoSc78} e bastante estudada consiste em determinar o \emph{número size-Ramsey} de um grafo $H$, denotado por $\hat{r}(H)$, sendo este o menor número de arestas tal que existe um grafo $G$ com essa quantidade de arestas de modo que $G \rightarrow H$. 
A princípio, Erd\H{o}s \cite{erdos81:_i} perguntou se o número size-Ramsey de caminhos seria tal que $\hat{r}(P_n) = \omega(n)$ e $\hat{r}(P_n) = o(n^2)$.
Em resposta, Beck \cite{Be83} forneceu uma construção probabilística de que o número size-Ramsey de caminhos é linear, i.e., $\hat{r}(P_n) = \Theta(n)$.  
O número size-Ramsey tem sido estimado para diversas classes de grafos tais como ciclos \cite{HaKo95}, árvores \cite{dellamonica12:_ramsey}, grau limitado \cite{rodl2000size}, potências de hiperárvores \cite{letzter2021size} e grafos bipartidos completos \cite{conlon2023three}. 

Ao longo dos últimos anos, sob o repertório de técnicas probabilísticas, o grafo aleatório binomial $G(n,p)$ tem sido o grafo alvo de coloração de arestas em que propriedades tipo Ramsey são estudadas \cite{rodl1993lower, rodl1995threshold}, especialmente quanto a funções limiares. 
Aqui, uma função limiar indica a probabilidade em que uma propriedade passa a ter alta probabilidade de ser satisfeita pelo $G(n,p)$ quando $n$ é suficientemente grande.
Uma variação do problema tipo Ramsey é a chamada \textit{propriedade anti-Ramsey} em que buscamos cópias \textit{rainbow} de um grafo $F$ nas colorações de $G(n,p)$, isto é, cópias de $F$ cujas arestas não repetem cor, denotamos esta propriedade por $G(n,p)\rbarrow F$.
Para cópias rainbow de $F$ em um grafo determinístico, o trabalho  de Fujita, Magnant e Ozeki~\cite{fujita2010rainbow} sumariza os resultados neste caso.

Nesta tese, vamos estimar a função limiar sobre a ocorrência de cópias rainbow para uma classe de grafos em $G(n,p)$. 
Esta variante foi primeiro estudada por Bohman, Frieze, Pikhurko e Smyth~\cite{bohman10:_aR_rgs} e depois por  Kohayakawa, Konstadinidis e Mota \cite{KoKoMo12} que adicionaram a condição de que as colorações de arestas são próprias.

 Outra frente precursora deste problema foi uma resposta à pergunta de Spencer presente em \cite{erdos79:_some}, que questionava sobre a existência de grafos com largura arbitrariamente grande conterem cópias rainbow de ciclos em colorações próprias. 
 Em 1992, Rödl e Tuza~\cite{roedl92:_rainb}
 provaram que para qualquer $\ell \geq 3$, existe uma probabilidade $p$ não nula tal que $G(n,p)\rbarrow C_\ell$ quase certamente.
 No entanto, a função limiar para a propriedade anti-Ramsey só passa a ser investigada com maior detalhe na última década.
 O trabalho de Kohayakawa, Konstadinidis e Mota~\cite{KoKoMo12}, com métodos de regularidade \cite{Kohayakawa97Szemeredi, KoRo03}  e de quasialetoriedade \cite{ChGr08} para grafos esparsos, estabeleceu que a  função limiar de $G(n,p)\rbarrow F$, com $F$ sendo um grafo fixado, possui limitante superior dependente da 2-densidade máxima de $F$.
 O foco desse problema se voltou então para a verificação do limite inferior desta propriedade, pois isto poderia determinar a função limiar ou ao menos gerar uma estimativa mais precisa da mesma.

Nesta direção, encontrar famílias de grafos em que a função limiar seja dada ou não pela 2-densidade do grafo se tornou um campo de investigação. 
Por exemplo, foi provado para ciclos a partir de cinco vértices que a função limiar segue da 2-densidade máxima de ciclos \cite{barros2021anti, NePeSkSt14}, 
como também para grafos completos \cite{kohayakawa2019anti, NePeSkSt14} com mais de cinco vértices.
Por outro lado, para o ciclo $C_4$, o grafo completo $K_4$ e algumas famílias infinitas de grafos obteve-se uma função limiar abaixo da 2-densidade \cite{mota2017advances, kohayakawa2019anti, KoKoMo16+, araujo2022anti}.

Destacamos, agora, a variante de problema de determinar o chamado \emph{número de Ramsey canônico},
 pois ele se coloca com resultados entre o Ramsey clássico e a propriedade anti-Ramsey mencionados anteriormente.
 Nesta variante, o número de cores disponíveis para colorir as arestas do grafo completo não é limitado.
 Um célebre resultado de Erd\H{o}s e Rado~\cite{erdoes_rado50}, de 1950, cujo teorema herda o nome de seus criadores, mostra que para um número inteiro $N$ suficientemente grande
e dada uma pré-ordem nos vértices de $K_N$, tem-se que qualquer coloração de arestas de $K_N$ contém um subgrafo completo $K_t$
em que um dos seguintes casos ocorre:
\begin{enumerate}[(a)]
    \item \label{it:canrain} $K_t$ é rainbow;
    \item \label{it:canmono} $K_t$ é monocromático;
    \item \label{it:cancan} duas arestas de $K_t$ só recebem a mesma cor se, e somente se, são incidentes ao mesmo vértice de maior (menor) valor na pré-ordem de $K_N$.
\end{enumerate}

A coloração que é dependente da pré-ordem dos vértices do grafo alvo de coloração, descrita no item (\ref{it:cancan}), é denominada de \emph{lexicográfica}. 
Dado um inteiro positivo $t$, o número de Ramsey canônico, $er(t)$, é justamente o menor inteiro $N$ tal que a afirmação acima vale.  
O ``Teorema de Erd\H{o}s e Rado'' é notório por elucidar que as colorações de arestas do grafo completo, mesmo com uma quantidade arbitrária de cores, podem gerar basicamente três tipos de coloração padrão para um subgrafo: monocromática, rainbow ou lexicográfica. 
Os melhores limitantes sobre $er(t)$ foram dados por Lefmann e R{\"o}dl~\cite{lefmann1995erdHos}, os autores provaram que

$$2^{c_1t^2}\leq er(t) \leq 2^{c_2t^2\log t},$$ 
onde $c_1$ e $c_2$ são constantes.

Podemos generalizar a coloração descrita em (\ref{it:canrain})-(\ref{it:cancan}) para encontrar outros grafos além de $K_t$, esta coloração é comumente chamada de \emph{canônica} \cite{lefmann1993canonical, axenovich2005canonical}.
Recentemente, Kam{\v{c}}ev e Schacht  \cite{kamvcev2023canonical} trouxeram um resultado semelhante ao de Erd\H{o}s e Rado para uma versão aleatória do problema, eles computaram a função limiar para a propriedade de colorações canônicas no grafo aleatório $G(n,p)$ conter cópias de $K_t$.

O principal resultado desta tese trata da propriedade anti-Ramsey de colorações próprias de arestas que contenham determinados subgrafos sem repetições de cores, a saber, colorações que evitem cópias rainbow do grafo bipartido completo.
Em essência, lidamos com o problema de encontrar famílias de grafos cuja função limiar dependa ou não de sua 2-densidade máxima, e exibimos em que condições o grafo bipartido completo apresenta limiar dado por sua 2-densidade.

Nossa principal ferramenta para provar o limiar para esta propriedade anti-Ramsey se baseia na técnica de 
Nenadov, Person, Škorić e Steger~\cite{NePeSkSt14}.
Tal técnica consiste em reduzir o problema de grafos aleatórios para um problema determinístico em que analisamos como as cópias do grafo bipartido completo podem ocorrer em um grafo com densidade limitada, e este último grafo é que deve ser cuidadosamente colorido com relação às cópias rainbow de um bipartido completo. 
Esta técnica também foi, com devidas adaptações, utilizada para provar o limiar de ciclos \cite{barros2021anti, NePeSkSt14} e grafos completos \cite{kohayakawa2019anti, NePeSkSt14}, porém em nossa demonstração conseguimos tratar grafos bipartidos completos que não são necessariamente grafos simétricos, nem regulares.  


Um problema secundário que lidamos durante a pesquisa foi introduzido por Conlon e Tyomkyn em \cite{conlontyomkyn}
e consiste de, dados os inteiros positivos $k$ e $n$, e um grafo $H$, deseja-se determinar o menor números de cores, escrito como $f_k(n,H)$, 
tal que se possa colorir propriamente as arestas do grafo completo $K_n$ evitando $k$ repetições de cópias de $H$ nas mesmas cores e vértice-disjuntas.
O mesmo trabalho se dedicou em provar resultados assintóticos de $f_k(n,H)$ para diferentes famílias de grafos no parâmetro $H$: ciclos, florestas, bipartidos e não-bipartidos.
Uma das questões deixadas por eles 
 foi determinar exatamente esta quantidade de cores para $n$ par, $k=2$ e $H = K_3$, ou seja, calcular $f_2(n, K_3)$ é evitar duas cópias isomorfas em cores de triângulos em colorações do $K_n$, onde $n$ é par.
 Investigamos isto para os primeiros valores pares de $n$ exibindo colorações, e também por via de uma construção algébrica de uma coloração que cobre infinitos valores pares de $n$.
 Resultados de colorações próprias de arestas de $K_n$, quando estas são representadas na matriz de adjacências, possuem aplicações diretas em quadrados latinos simétricos e, estes últimos por sua vez se refletem em quasigrupos \cite{andersen1994symmetric}.
 
Apesar do trabalho de Conlon e Tyomkyn ser recente, esse parâmetro introduzido por eles sobre padrões de repetições em colorações do grafo completo já vem sendo estudado para repetições de grafos tais como ciclos pares \cite{janzer2023rainbow}, potências de árvores \cite{liu2022color} e subdivisões do grafo completo \cite{xu2022color}.

Em suma, neste trabalho 
%% tem como objetivo o estudo de diversas propriedades de grafos que garantam a existência de estruturas monocromáticas e multicoloridas em grafos cujas arestas estão coloridas.
são investigadas propriedades dos tipos Ramsey e anti-Ramsey
de grafos aleatórios e determinísticos em diferentes versões. 
Lidamos com problemas cujo ponto em comum é a coloração própria de arestas sob a restrição de encontrarmos certas cópias coloridas de grafos específicos.



\section*{Organização do texto}


Iniciamos no Capítulo \ref{cap:prelim} apresentando brevemente os conceitos mais relevantes sobre grafos neste trabalho.
O Capítulo \ref{cap:triangles} exibe um resultado de padrões em coloração de arestas do grafo completo. 
Especificamente, dado um natural $n$, calculamos a menor 
quantidade de cores em que existe uma coloração própria de $E(K_n)$ de modo a se evitar a formação de triângulos disjuntos em vértices que repetem a trinca de cores nas arestas.
Esse número era desconhecido quando $n$ é par e foi determinado para os primeiros números pares (ver Tabela \ref{tab:valores}) e também para uma família infinita de valores pares de $n$ (Teorema \ref{teo:seq-f2}) via  construção direta de colorações.

%%%% seção conclusão?
%%%A importância e vigor da pesquisa em torno dos tópicos que propomos podem ser observados nos trabalhos clássicos de diversos pesquisadores renomados, como Erd\H{o}s, Rödl, Szemerédi, e Tao, dentre outros, e nos trabalhos modernos de proeminentes pesquisadores como Conlon, Gowers, e Schacht. 
%%% Os trabalhos desses autores, que muitas vezes consideram grafos aleatórios, mostram que ainda existe muito a ser feito no desenvolvimento de uma teoria sólida para entendermos a existência de uma certa ordem em estruturas aparentemente caóticas, o cerne da Teoria de Ramsey.

 O Capítulo \ref{cap:anti} traz como resultado inédito as condições para que a propriedade anti-Ramsey ocorra no grafo aleatório $G(n,p)$ com relação a conter cópias isomorfas ao grafo bipartido completo $\K$. 
A saber, dados inteiros $\ell,r \geq 3$, provamos que a função limiar para a propriedade do grafo aleatório ser colorido evitando cópias rainbow de $\K$ depende da 2-densidade máxima de $\K$, especificamente, a função limiar encontrada é 
$n^{(\ell+r-2)/(\ell r-1)}$, e isto é consequência dos Teoremas \ref{teo:conditional} e \ref{teo:mainrb}.
Também verificamos que tal função limiar é assintoticamente menor do que este valor se $\ell=2$ e $r\geq 2$ (vide Corolário \ref{cor:k2r}).
%%O Apêndice \ref{cap:app}

Por último, o Capítulo \ref{cap:conclusao} encerra nossa discussão com comentários finais a respeito dos resultados destacando contribuições e próximos passos para os problemas investigados. 

\chapter[Conceitos preliminares]{Conceitos preliminares}
\label{cap:prelim}

Neste capítulo introduzimos as notações e conceitos básicos utilizados ao longo desta tese, seguimos notação e terminologia padronizadas (veja, por exemplo, os livros \cite{diestel10:_graph_theor, janson00:_random_graph}).

Dadas funções $f, g : \mathbb{N} \rightarrow \mathbb{R}_+$ e um natural $n$, dizemos que $f(n) \ll g(n)$
se e somente se, para
todo $\epsilon > 0$, existe $n_0$ tal que, se $n \geq n_0$, então $f(n) < g(n)$.

Um \textit{grafo} $G$ é um par de conjuntos $(V,E)$, também denotados por $V(G)$ e $E(G)$, os quais são
os conjuntos dos \textit{vértices} e das \textit{arestas} de $G$, respectivamente.
Dizemos que um grafo $H$ é um \textit{subgrafo} de $G$
se $V(H) \subseteq V(G)$ e $E(H) \subseteq E(G)$. 
Por vezes, dados grafos $G$ e $H$, se existe um subgrafo $F\subseteq G$ que é isomorfo a $H$, dizemos que $F$ é \textit{cópia} de $H$ em $G$. 

Duas classes de grafos bastante presentes nos Capítulos \ref{cap:triangles} e \ref{cap:anti} são os ciclos e os grafos bipartidos completos, respectivamente.
Um grafo $C$ é um \textit{ciclo} de comprimento $n$ se ${V(C)=\{v_1, \ldots, v_n\}}$ e $E(C) = \{v_1v_2, v_2v_3,\ldots, v_{n-1}v_n, v_nv_1\}$.
Se $G$ é um grafo tal que $V(G)$ pode ser dividido em dois subconjuntos disjuntos, $A$ e $B$, e cada
aresta de $E(G)$ possui extremidades entre $A$ e $B$, então dizemos que $G$ é um grafo \textit{bipartido}. 
Ademais, se cada vértice de $A$ é adjacente a cada vértice de $B$, chamamos $G$ de \textit{bipartido completo}, denotando-o por $G[A,B]$, onde chamamos os conjuntos $A$ e $B$ de \textit{partes} de $G$.

Neste trabalho estudamos problemas de coloração de arestas. 
Definimos formalmente, dado um grafo $G$, uma \textit{coloração de arestas} de $G$ é uma função $\chi: E(G) \rightarrow \mathbb{N}$. 
A coloração $\chi$ de $G$ é dita ser \textit{própria} se as arestas adjacentes possuem cores distintas, i.e., para cada vértice $v \in V(G)$, tem-se $\chi(\{v,u\}) \neq \chi(\{v,w\})$ para cada par $u,w$ de vizinhos de $v$.
Chamamos de \textit{índice cromático} de $G$, denotado por $\chi'(G)$, o número mínimo de cores tal que existe uma coloração própria de $E(G)$.

O \textit{grafo aleatório binomial} no modelo de Erd\H{o}s e Rényi, $G \in G(n, p)$, é um grafo com um conjunto de $n$ vértices, em que entre quaisquer dois vértices existe uma aresta com probabilidade $p$ e essas arestas são independentes.

Ressaltamos uma definição importante que é objeto de investigação no grafo aleatório no Capítulo \ref{cap:anti}.
Uma função $\hat{p}: \mathbb{N} \rightarrow [0,1]$ é chamada de \textit{função limiar} para uma propriedade
$\cP$ no $G(n,p)$ se 
\[
  \lim_{n\to\infty}\mathbb{P}\left(G(n,p) \in \cP\right)
  =
  \begin{cases}0&\text{se }p \ll \hat p, \\
    1&\text{se }p \gg \hat p.
  \end{cases}
\]
Os limites na definição acima de função limiar são chamados de $0$-\emph{afirmação} e $1$-\emph{afirmação}, respectivamente. 

Uma vez que os limiares são determinados por ordem de magnitude, vamos nos referir aos limiares $\hat{p}$ como \emph{o limiar} da propriedade $\cP$.

Um conjunto de grafos $\mathcal{A}$ é \textit{monótono} se para qualquer $H \in \mathcal{A}$ e $H \subseteq G$, tem-se que $G \in \mathcal{A}$. 

Muitos resultados conhecidos de propriedades extremais no grafo aleatório dependem das densidades de um grafo.
A \emph{$2$-densidade máxima} de um grafo $H$, denotada por~$m_2(H)$, é definida como 
\begin{equation*}
  m_2(H)=\max\left\{\frac{|E(J)|-1}{|V(J)|-2}\colon J\subset H,\;|V(J)|\geq3\right\},
\end{equation*}
onde assumimos~$|V(H)|\geq3$, e 
\[
    m(H) = \max\left\{\frac{|E(J)|}{|V(J)|}\colon J\subset H,\;|V(J)|\geq1\right\},
\]
é a \emph{densidade máxima} de $H$, a qual frequentemente vamos chamar só de densidade de $H$.

Um resultado clássico de Bollobás que relaciona diferentes conceitos deste capítulo é dado a seguir.

%%\begin{teorema}[\cite{friedgut96:_every}]\label{teo:monotona}     Toda propriedade monótona não trivial de grafos possui uma função limiar. \end{teorema}

\begin{teorema}[\cite{bollobas1981threshold}]\label{teo:gnpcontem}
    Seja $H$ um grafo fixo. Então, $\hat{p} = n^{1/m(H)}$ é o limiar para a propriedade de que $G(n,p)$ contém $H$.
\end{teorema}

Um grafo $H$ é dito 2-\textit{balanceado} se a 2-densidade máxima de $H$ é maior ou igual que a 2-densidade de qualquer um de seus subgrafos que contêm pelo menos três vértices. 
Dados $\ell,r\geq 2$ inteiros, o grafo bipartido completo $\K$ é um exemplo de grafo 2-balanceado.





\chapter[Colorações de grafos completos sem $K_3$ repetidos]{Colorações próprias de arestas de grafos completos sem triângulos repetidos}
\label{cap:triangles} 


%% Nesta seção \footnote{Os resultados descritos nesta seção foram desenvolvidos em um projeto conjunto com Fábio Botler, Lucas Colucci, Guilherme Mota, Roberto Parente e Matheus Secco.}, vamos considerar grafos finitos e simples. 
Os resultados descritos neste capítulo foram desenvolvidos em um projeto conjunto com Fábio Botler, Lucas Colucci, Guilherme Mota, Roberto Parente e Matheus Secco \cite{botler2022proper}.
Vamos considerar grafos finitos e simples. 
Recordamos que uma coloração de arestas é chamada de \emph{própria} se para cada par de arestas que compartilham um vértice em comum as cores são distintas. Um artigo recente de Conlon e Tyomkyn  \cite{conlontyomkyn} introduziu um novo conceito relacionado a colorações próprias, definido a seguir: 

\begin{defi}\label{def:fkn}
            Para $k,n\geq 2$ e fixado um grafo $H$, chamamos de $f_k(n,H)$ o menor inteiro $C$ tal que existe uma coloração
            própria de arestas de $K_n$ com $C$ cores sem conter $k$ cópias isomorfas em cores (\emph{repetições}) de $H$ disjuntas em vértices. 
\end{defi}

  \begin{center}
			\includegraphics[scale=0.25, keepaspectratio=true]{K6-repet-mark.png}
			\captionof{figure}{Uma coloração própria do $K_{6}$ identificando triângulos repetidos.} 
			\label{fig:K6} 
	\end{center}
	

Para $H$ sendo um ciclo par, Conlon e Tyomkyn \cite{conlontyomkyn} perguntaram se para todo $\eps >0$,
existe um $t_0 = t_0(\eps)$ tal que, para todo 
$t \geq t_0$, $f_2(n, C_{2t}) = \Omega(2^{2-\eps})$.
%
Xu, Zhang, Jing e Ge~\cite{ge2020color} provaram que $f_3(n, C_4) = \Theta(n)$ e depois refinaram a conjectura de Conlon e Tyomkym questionando se
$f_2(n, C_{2t}) = \Omega(n^{2-2/t})$. 
Por fim, Janzer \cite{janzer2023rainbow} resolve esse problema provando o seguinte resultado:
 
    \begin{teorema}[\cite{janzer2023rainbow}]\label{teo:janzer-ciclos}
           Sejam $k,t \geq 2$ inteiros fixados. Então $f_k(n,C_{2t}) = \Omega\big(n^{\frac{k}{k-1} \cdot\frac{t-1}{t}}\big)$.
    \end{teorema}

Nós estudamos o caso em que os parâmetros são $k = 2$ e $H = K_3$ na Definição \ref{def:fkn}. 
No artigo inicial de Conlon e Tyomkyn foi mostrado que $f_2(n,K_3) = n$ para
$n$ ímpar, e isso implica que $f_2(n,K_3) \leq n + 1$ para $n$ par. 
Além disso, como toda coloração própria de $K_n$ requer ao menos $n-1$ cores, segue que $f_2(n,K_3) \geq n- 1$.
Portanto, o problema que permanecera em aberto é decidir, para $n$ par, se $f_2(n,K_3)$ é igual a $n-1$, $n$ ou $n + 1$.

\section{Casos pequenos}

Nesta subseção, calculamos $f_2(n,K_3)$ para $n \in \{4,6,8,12\}$, o caso $n=10$ é coberto por um resultado apresentado na subseção seguinte. 
É evidente que $f_2(4,K_3) = 3$, dado que não existem pares de triângulos vértices-disjuntos em $K_4$, e com isso, busca-se simplesmente por qualquer coloração própria. 
Os próximos dois resultados mostram que, para $n = 6$ e $n = 8$, exatamente $n + 1$ cores são necessárias.

Chamamos dois triângulos vértices-disjuntos de \textit{similares} em uma coloração de um grafo se suas arestas recebem as mesmas três cores;  uma cor é dita ser \textit{cheia} se as arestas com tal cor formam um matching perfeito no grafo completo. 
Com essas breves definições podemos provar os resultados nos casos pequenos.

\begin{teorema}\label{teo:n6K3}
       $f_2(6,K_3)=7$.
\end{teorema}
\begin{proof}
    Das considerações anteriores, é suficiente provar que 
    toda coloração de $K_6$ com no máximo seis cores contém um par de triângulos disjuntos em vértices que são \emph{similares} (i.e., triângulos cujas arestas recebem o mesmo conjunto de cores). 
   Suponha que isso não ocorre, e seja $V(K_6) =\{1,\dots,6\}$.
   Como $K_6$ tem $15$ arestas, e  cada uma dentre as seis classes de cor consiste de, no máximo, três arestas, tem-se que pelo menos três cores geram um emparelhamento perfeito (são cores \emph{cheias}). 
   Ademais, a união de quaisquer dois emparelhamentos perfeitos induzem um ciclo de comprimento seis.
   Supomos sem perda de generalidade que o ciclo $123456$ está colorido alternando as cores $1$ e $2$. 
   Nós chamamos uma aresta de \emph{corda curta} se ela conecta dois vértices com distância $2$ neste ciclo (em particular, uma corda curta forma um triângulo com arestas de cores $1$ e $2$). 
   É fácil ver que se temos três cordas curtas de mesma cor, então obtemos dois triângulos similares distintos. 
   Além disso, duas cordas curtas das mesmas cores devem ser da forma $i(i+2)$ e $(i+1)(i+3)$ para algum $i$ (índices mod $6$), já que de outra maneira elas gerariam dois triângulos disjuntos similares.
    
    Agora, suponha que existe alguma corda curta colorida com uma cor cheia, digamos que, a aresta $13$ está colorida com a cor $3$. É direto checar que as arestas restantes da cor $3$ devem ser $25$ e $46$, que é uma contradição, uma vez que $123$ e $456$ seriam triângulos similares.  
    
    Logo, nenhuma corda curta pertence a uma cor cheia. Como há $6$ cordas curtas e cada uma (de até três) cores que não são cheias pode cobrir, no máximo, duas cordas curtas, nós concluímos que toda cor que não é cheia aparece em exatamente duas cordas curtas. Sem perda de generalidade, supomos que $13$ e $24$ recebem a cor $4$, $35$ e $46$ a cor $5$, e $51$ e $62$ a cor $5$. Neste caso, os triângulos $135$ e $246$ são similares, o que é uma contradição.
\end{proof}


\begin{teorema}\label{teo:n8K3}
     $f_2(8,K_3)=9$.
\end{teorema}
\begin{proof}
Pelas considerações anteriores, é suficiente provar que toda coloração do $K_8$ com no máximo $8$ cores  contém um par de triângulos similares disjuntos em vértices (i.e., triângulos cujas arestas recebem o mesmo conjunto de cores).
Suponha que não é esse o caso, e seja $\{1,\dots,8\}$ o conjunto de vértices do $K_8$. 
Como  $K_8$ tem $28$ arestas, e cada uma dentre as, no máximo, oito classes de cores consiste de até quatro  arestas. Segue-se que pelo menos quatro cores geram um emparelhamento perfeito (chamamos tal cor de \emph{cheia}). Dividimos a prova em dois casos.

\medskip
\textbf{Caso 1:} Existem duas classes de cores cuja união induz um $C_8$. 

Sem perda de generalidade, digamos que as cores $1$ e $2$ geram o ciclo $12345678$ (de modo que a aresta $12$ tenha a cor $1$). 

Se uma corda curta, digamos, $82$, pertence a uma cor cheia (por simplicidade, digamos que seja na cor $3$), então é direto checar que as outras arestas de cor $3$ serão $15$, $36$ e $47$, caso contrário, existem dois triângulos disjuntos de cores $123$. 
Agora considere a corda curta $46$ colorida, digamos, com a cor $4$. 
Tanto a corda $35$ quanto a $57$ está colorida com uma nova cor, que seja a $5$. Suponhamos que $57$ seja colorida com $5$. 
Agora, vê-se que não há aresta incidente ao vértice $2$ colorida pela cor $4$ (caso contrário, existiriam dois triângulos disjuntos de cores $1$, $3$ e $4$). 
Como temos até oito cores, concluímos que cada cor, exceto a $4$, deve aparecer em alguma aresta incidente ao vértice $2$. 
Em particular, alguma aresta deve ser colorida com a cor $5$, e novamente é simples checar que isto é uma contradição, pois criaria dois triângulos disjuntos de cores ou $1$, $2$ e $5$ ou $2$, $3$ e $5$.

Logo, podemos supor que toda corda curta pertence a uma cor que não é cheia.
Como temos até quatro cores assim e oito cordas curtas, concluímos que há exatamente quatro cores que não são cheias (cobrindo exatamente três arestas cada), onde cada cor cobre duas cordas curtas que devem ser vizinhas. 
Dito isto, suponhamos que as arestas $13$ e $24$, $35$ e $46$, $57$ e $68$, $71$ e $82$ são coloridas com as cores $3$, $4$, $5$ e $6$, respectivamente.
A terceira aresta de cor $3$ deve ser $58$ (pois as outras arestas ligando os vértices a $\{5,6,7,8\}$ já foram pintadas por outras cores), e, da mesma forma, $72$, $14$ e $36$ devem ser coloridas com as cores $4$, $5$ e $6$, respectamente. Os triângulos $134$ e $578$ são  similares (ambos com as cores $1$,$3$ e $5$), uma contradição.

\medskip
\textbf{Caso 2:} Todo par de cores cheias gera duas cópias de $C_4$.

Neste caso, é simples verificar que a união de três cores cheias, digamos, $1$, $2$ e $3$, devem gerar um cubo  tal que cada classe de cor corresponde a uma dimensão do poliedro. Uma quarta cor cheia, que seja a $4$, não pode cobrir as arestas internas de cada face do cubo, caso contrário criaria triângulos similares nas faces opostas.
Isto implica que as arestas de cor $4$ são, precisamente, as diagonais principais do cubo.

As cores $5$, $6$, $7$, $8$ devem cobrir as diagonais das faces do cubo. Evidentemente, dentro da face, as duas diagonais devem ter cores diferentes (caso contrário a face oposta forma um triângulo similar).
Além disso, se uma face tem diagonais de cores $5$ e $6$, a oposta  deve ter outas duas cores, i.e., $7$ e $8$, evitando criar um triângulo. 
Mais que isso, duas faces adjacentes não podem ter o mesmo par de cores em suas diagonais, uma vez que isso geraria dois triângulos similares juntamente com as diagonais principais do cubo.

As observações acima implicam cada um dos $\binom{4}{2}=6$ pares de cores $5,6,7,8$ aparece nas diagonais de, exatamente, uma das faces do cubo.
Considere as três faces que contêm uma diagonal de cor $5$.
Nenhuma dessas duas faces pode ser oposta, e, com isso, as três são mutualmente vizinhas. Isto é uma contradição, uma vez que não há três diagonais independentes, duas a duas, em três faces mutuamente adjacentes do cubo.
\end{proof}

Por fim, com o auxílio de um computador, checamos que $f_2(12,K_3) = 12$ (a Figura \ref{fig:K12} exibe a coloração), que é o menor número par $n$ tal que $f_2(n,K_3) = n$. Isto mostra que cada valor em $\{n - 1, n, n + 1\}$ pode ocorrer para $f_2(n,K_3)$ com $n$ par.
A técnica usada foi uma redução ao problema SAT, que foi gerado pelo Sage \cite{sagemath} e resolvido por um SAT solver \cite{lingeling}. 
O código está disponível em   %
%\texttt{https://github.com/robertoparente/k3copy\_sat}.
\texttt{https://github.com/paulo-matias/k3copy\_sat}.

  \begin{center}
			\includegraphics[scale=0.7, keepaspectratio=true]{k12-12colors.jpeg}
			\captionof{figure}{Uma coloração do $K_{12}$ com 12 cores.} 
			\label{fig:K12} 
	\end{center}
	
	\section{Uma família infinita de valores de $n$}
	Nesta subseção, calculamos o valor exato de $f_2(n, K_3)$ para uma família infinita de $n$ pares, que é o primeiro resultado mais abrangente conhecido para o problema.

  \begin{teorema}\label{teo:seq-f2}
        Dado $t$ um inteiro positivo, se $n=3^t + 1$, então $f_2(n,K_3) = n-1$.
   \end{teorema}
    \begin{proof}
        Vamos construir uma coloração própria de arestas de $K_n$ sob a condição de $n=3^t+1$ e mostrar que tal coloração não possui triângulos repetidos disjuntos em vértices.
        
        Destaquemos um vértice $u \in K_n$ de maneira que todos os  $3^t$ vértices restantes são elementos de  $\ZZ_3^t = \{x=(x_1, \dots, x_t) \colon x_i \in \ZZ_3, \ \forall i \in [t]\}$.
        
        Definimos a seguinte coloração $\chi \colon E(K_n) \rightarrow \{0,1,2\}^t = \ZZ_3^t$:
        \[
            \chi(\{x,y\}) = 
                \left\{\begin{matrix}
                (2x_1, \cdots, 2x_t) \mod 3, & \text{ se } y = u \\
                (x_1+y_1, \cdots, x_t+y_t) \mod 3, & \text{ se } u \notin \{x,y\},
                \end{matrix}\right.
        \]
        onde operamos a congruência % $\mod 3$
        a cada componente da $t$-tupla.
        
        A coloração $\chi$ obviamente utiliza até $n = 3^t$ cores pela construção sobre $\ZZ_3^t$. 
        Notemos também que $\chi$ é uma coloração própria.
        De fato, se duas arestas adjacentes e diferentes $e= \{x,y\}$ e $f=\{x,z\}$ recebem a mesma cor por $\chi$, então
        \begin{itemize}
            \item \textbf{Caso 1:} se $x = u$, então teríamos $2y_i = 2z_i$ para cada $i \in [t]$, logo, $y = z$, uma contradição;
            \item \textbf{Caso 2:} se $x \neq u$, então teríamos que $x_i + y_i = x_i + z_i$ (quando $y$ e $z$ são ambos diferentes de $u$) para cada $i \in [t]$, ou teríamos que $2x_i = x_i+z_i$ (se, s.p.g., $y = u$) para cada $i \in [t]$.
            No primeiro caso, obteríamos $y =z$, no segundo por sua vez que $x = z$, e com isso, obtemos outra contradição.
        \end{itemize}
        Resta-nos provar que $\chi$ não gera triângulos disjuntos repetidos em cores.
        Vamos supor, por contradição, que existem dois triângulos disjuntos com o mesmo conjunto de cores.
        
        Primeiro, suponha que ambos os triângulos não contém o vértice $u$ e os chamemos de $\{x,y,z\}$ e $\{a,b,c\}$, respectivamente, de cores $\alpha, \beta, \gamma$. Pela definição de $\chi$ podemos escrever para cada $i \in [t]$ que
        \[
            \left\{\begin{matrix}
            \alpha_i = a_i +b_i\\ 
            \beta_i = b_i+c_i\\ 
            \gamma_i = c_i+a_i
            \end{matrix}\right.
            \implies a_i = -\alpha_i+\beta_i-\gamma_i.
        \]
        De maneira análoga, $b_i$ e $c_i$ são determinados. 
        Resolvendo o mesmo sistema de equações para $x_i, y_i$ e $z_i$,
        concluímos que $a_i = x_i$ para cada $i \in [t]$.
        Isto implica que $a = x$ e contradiz que os triângulos são disjuntos.
        
        Agora, suponha que algum dos triângulos repetidos contém $u$, a saber, são $\{x,y,u\}$ e $\{a,b,c\}$ de cores $\alpha, \beta, \gamma$. 
        Disso temos para cada $i \in [t]$ que
        \[
            \left\{\begin{matrix}
            \alpha_i = x_i+y_i = a_i +b_i\\ 
            \beta_i = 2y_i = b_i+c_i\\ 
            \gamma_i = 2x_i = c_i+a_i
            \end{matrix}\right.
            \implies 
            \left\{\begin{matrix}
            2a_i = \alpha_i - \beta_i + \gamma_i\\ 
            2y_i = \alpha_i - \beta_i + \gamma_i
            \end{matrix}\right.
        \]
        que em $\ZZ_3$ implica em $y_i = a_i$, para cada componente $i$. Logo, $a = y$ contradizendo que os vértices da repetição  são disjuntos.
    \end{proof}
    
    \section{Problemas em aberto}
\label{subsec:repeated-prob}

        O parâmetro $f_2(n, K_3)$ com $n$ par continua a ser não compreendido sobre a existência de mais algum padrão em $\{n-1,n,n+1\}$ com outras sequências numéricas (Tabela \ref{tab:valores}).
        Variar o número de cópias de outros e também os grafos fixados forma um campo aberto a ser explorado.
        Xu e Ge \cite{xu2022color} provaram o seguinte limite inferior: $f_2(n, H_t) = \Omega(n^{1+\frac{1}{2t-3}})$, onde $H_t$ é uma 1-subdivisão do grafo completo $K_t$. Eles questionam se existe um expoente $\beta$ tal que $f_2(n, H_t) = \Theta(n^\beta)$.
        
        Janzer \cite{janzer2023rainbow} provou um resultado mais geral sobre repetições de ciclos pares, que em particular diz  $f_2(n, C_4) = \Omega(n)$, enquanto Conlon e Tyomkin \cite{conlontyomkyn} verificaram que $f_2(n, C_4) = O(n^{3/2})$. Resta-nos a seguinte pergunta:
        
        \begin{questao}\label{q:repeated-C4}
            Dado $n$ natural, $f_2(n, C_4) = \Theta(n)$?
        \end{questao}

  \begin{table}[] 
    \centering
        \begin{tabular}{|l|lllllllllll}
        \hline
        $n$ & 4 & 6 & 8 & \text{\color{blue}10} & 12 & 14 & $\cdots$ & 26 & \text{\color{blue}28} & 30 & $\cdots$ \\ \hline
        $f_2(n, K_3)$ & 3 & 7 & 9 & \text{\color{blue}9}  & 12 & \text{\color{red}?}  & $\cdots$ & \text{\color{red}?}  & \text{\color{blue}27} & \text{\color{red}?}  & $\cdots$ \\ \hline
        \end{tabular}
        \caption{Valores de $f_2(n,K_3)$, em azul seguem do Teorema \ref{teo:seq-f2}.}\label{tab:valores}
\end{table}
 
 Conlon e Tyomkin \cite{conlontyomkyn} provaram que para árvores $T$ com $m$ arestas, existe alguma constante $k_0 = k_0(T)$ tal que para todo $k \geq k_0$ vale $f_k(n,T) = \Theta(n^{\frac{m+1}{m}})$. Os autores deixam a seguinte questão:
 
 \begin{questao}[\cite{conlontyomkyn}]\label{q:repeated-T}
     Qual o menor valor de $k_0$ tal que para todo inteiro $k \geq k_0$, vale $f_k(n,T) = \Theta(n^{\frac{m+1}{m}})$? 
 \end{questao}
 
 Um grafo $r$-partido completo é um grafo $r$-partido em que existe uma aresta para cada par de vértices de diferentes conjuntos independentes;
 se existem $n_1, \ldots, n_r$ vértices nos $r$ conjuntos independentes, denotamos o grafo $r$-partido completo por $K_{n_1, \ldots, n_r}$.
 Uma primeira generalização do problema de computar o parâmetro $f_k(n,H)$ é mudar o grafo base de coloração de $K_n$ para $K_{{\underbrace{n, \ldots, n}_{\text{$r$ vezes}}}}$, escrevemos o novo parâmetro por $f_k^r(n,H)$.

Determinar o número $f_k^r(n, H)$ é de fato um problema extremal, e tal como $f_k(n,H)$, o parâmetro é monótono crescente em $n$, e decrescente com relação ao número $k$ e ao grafo $H$ quando tomamos subgrafos que contém $H$. 

A mudança do grafo base de coloração nos dá que 
%
    \begin{equation}\label{eq:frkn}
        \chi'(K^r_n) \leq f_k^r(n, H) \leq f_k(rn, H),
    \end{equation}
%
onde $\chi'(K^r_n)$ é o índice cromático do $r$-partido completo. 
Logo, quando $r$ é uma constante, os resultados sobre os limitantes superiores de $f_k(n,H)$ são mantidos para $f_k^r(n,H)$ a menos de uma constante dependente de $r$.
Além disso, se o grafo $H$ contém uma clique $K_{r+1}$, então ${f_k^r(n, H) = \chi'(K^r_n)}$, pois $K^r_n$ não contém $K_{r+1}$ e assim, toda coloração própria de $K_n^r$ evita repetições de $H$.

É direto que $f_k(n,H) = f_k^1(n,H)$ e também que $f_k^2(n, K_3)= n$, este último porque grafos bipartidos não contêm triângulos e a coloração própria de $E(K_{n,n})$ utiliza $n$ cores. 
A questão inicial levantada por nós é analisar o mesmo padrão de repetição de triângulos começando pelo grafo tripartido completo, ou seja, estimar o número $f_2^3(n, K_3)$, visto que neste grafo base a presença de triângulos se dá com cada vértice do triângulo em uma parte distinta:

\begin{questao}\label{q:tripartido}
    Dado $n$ natural, qual o valor de $f_2^3(n, K_3)$?
\end{questao}

Agora, voltados para este caso em que $H=K_3$, utilizando as propriedades apresentadas,
a princípio, é fácil refinar o limite superior em (\ref{eq:frkn}) para $f_2^3(n,K_3) \leq f_2(3n, K_3)-1$, quando $n \geq 4$ é par.
Novamente, a principal dificuldade do problema está em determinar os valores de $f_2^3(n, K_3)$ com precisão, visto que seu comportamento assintótico segue o mesmo de $f_2(N, K_3)$, para algum $N$ inteiro.

%% De casos iniciais, temos que $f_2^3(2, K_3) = 6$ (PROVA), $f_2^3(3,K_3) \in \{8,9\}$.
A seguir, mostramos que $f_2^3(2, K_3) = f_2(6,K_3) - 1= 6$.

\begin{fato}
    $f_2^3(2, K_3) = 6$.
\end{fato}
\begin{proof}
    Para uma coloração de arestas do tripartido completo $K_{2,2,2}$ utilizando 4 cores, 
    %chamemos as cores pelos elementos do conjunto $[4]$, 
    como tal grafo possui 12 arestas, cada classe de cor forma um emparelhamento perfeito e esta coloração é única, a menos de permutações de cores, para ser própria (Figura \ref{fig:K222-5c}). 
    Note que os triângulos disjuntos contidos nesta coloração são repetidos, a saber, $K_{2,2,2}$ possui 8 triângulos e cada um deles possui uma repetição vértice-disjunta formando quatro pares distintos de triângulos repetidos: 123, 124, 134, 234. 
        
    Partindo desta coloração, se usássemos uma quinta cor há dois casos: esta poderia ser aplicada em duas arestas da mesma classe de cor ou em duas arestas de classes distintas. 
    No primeiro caso, existe um triângulo que não utiliza a cor substituída e continuaríamos com sua repetição vértice-disjunta.
    No segundo caso, basta verificarmos que um dos quatro pares de triângulos vértice-disjuntos não é afetado pela recoloração de duas arestas, o que pela coloração inicial forma uma repetição de triângulos.
    % suponha que as arestas recoloridas pela nova cor tinham as cores 1 e 2. 
    % Com isso, existem dois vértices, digamos que $1$ e $2$, ainda incidentes em arestas de cores 1 e 2. Vamos ver que esses pares de arestas fecham triângulos repetidos com uma das cores que não foram trocadas. 
    Como no $K_{2,2,2}$ cada aresta está contida em dois triângulos distintos, tem-se que as duas arestas recoloridas estão em quatro triângulos, e não é difícil ver que dois destes triângulos são vértice-disjuntos. 
    Logo, os quatro triângulos contendo arestas recoloridas alteram até três dos quatro triângulos possíveis da coloração inicial de 4 cores.
    
    A Figura \ref{fig:K222-6c} exibe uma coloração própria do $K_{2,2,2}$ com 6 cores e sem triângulos repetidos que segue a coloração proposta no parágrafo anterior e acrescenta uma sexta cor para evitar a repetição não coberta pela quinta cor.
    
\end{proof}

 \begin{center}
			\includegraphics[scale=0.75, keepaspectratio=true]{K222-5c.pdf}
			\captionof{figure}{A única coloração própria do $K_{2,2,2}$ com 4 cores a menos de permutações de cores.} 
			\label{fig:K222-5c} 
	\end{center}
  \begin{center}
			\includegraphics[scale=0.75, keepaspectratio=true]{K222-6c.pdf}
			\captionof{figure}{Coloração própria do $K_{2,2,2}$ com 6 cores sem triângulos repetidos.} 
			\label{fig:K222-6c} 
	\end{center}

\chapter{Um problema do tipo anti-Ramsey em grafos bipartidos completos}\label{cap:anti}

Dados grafos $G$ e $H$ e uma coloração de arestas de $G$, dizemos que uma cópia de $H$ em $G$ é \emph{rainbow} se não existem duas arestas de $H$ com a mesma cor.
%
Dados grafos~$G$ e~$H$, estamos interessados na seguinte propriedade denotada por ${G\rbarrow H}$: para toda coloração \textit{própria} das arestas de~$G$ (com uma quantidade arbitrária de cores),
existe uma cópia rainbow de~$H$ em~$G$, i.e., um subgrafo de $G$ isomorfo a $H$ com todas as arestas de cores distintas.


A propriedade anti-Ramsey ${G\rbarrow H}$ é monótona, isto é, se ${G\rbarrow H}$, então vale ${F\rbarrow H}$, para todo subgrafo $F$ de $G$. Tal monotonicidade em 
grafos aleatórios implica que essa propriedade admite uma função limiar \cite{BoTh87}, que nós denotamos por $\hat{p}_H$. 


\section{Resultados existentes}

No trabalho~\cite{KoKoMo12}, o orientador e colaboradores provaram o seguinte resultado para grafos aleatórios: a propriedade anti-Ramsey $G(n,p)\rbarrow H$ vale assintoticamente quase certamente sempre que $p\gg n^{-1/m_2(H)}$ para todo grafo fixo~$H$. Em outras palavras, garante-se em termos da $2$-densidade máxima de um grafo $H$ um limitante superior para a função limiar $\hat{p}_H$, como está enunciado a seguir.

\begin{teorema}[%Kohayakawa, Konstadinidis e Mota~
\cite{KoKoMo12}]
    \label{teo:conditional}
         Seja $H$ um grafo fixo. Então existe uma constante $C>0$ tal que, para $p=p(n)\geq Cn^{-1/m_2(H)}$, assintoticamente quase certamente temos ${G(n,p)\rbarrow H}$. 
         Em particular, $\hat{p}_H \leq n^{-1/m_2(H)}$.
  \end{teorema}
    
Dado o Teorema \ref{teo:conditional} e sabendo que a $2$-densidade máxima tem um papel fundamental em Teoria de Ramsey, é natural perguntar se $n^{-1/m_2(H)}$ é função limiar para a propriedade $G(n,p)\rbarrow H$ para qualquer grafo $H$, o que exige provar o limitante inferior do limiar como $\hat{p}_H \geq n^{-1/m_2(H)}$.

No entanto, em outro trabalho \cite{KoKoMo16+}, o mesmo grupo de pesquisadores obteve uma resposta negativa para essa pergunta, eles exibiram uma família infinita de grafos cuja função limiar é inferior a $n^{-1/m_2(H)}$, 
e mais recentemente, o trabalho \cite{araujo2022anti} estendeu a família de grafos em que o limiar difere da 2-densidade.
Porém a 2-densidade máxima se relaciona, de fato, à função limiar para alguns grafos $H$, como foi provado por Nenadov,
Person, Škorić e Steger~\cite{NePeSkSt14}.
Estes últimos autores desenvolveram um \emph{framework} geral de prova para limitantes inferiores de diversos problemas tipo Ramsey em grafos aleatórios.

\begin{teorema}[%Nenadov, Person, Škorić e Steger~
\cite{NePeSkSt14}]
\label{teo:nen}
  Seja $H$ um ciclo de pelo menos $7$ vértices ou um grafo completo de pelo menos $19$ vértices. 
  Existe uma constante $c > 0$ tal que se
  $p=p(n)\leq cn^{-1/m_2(H)}$, 
  então com alta probabilidade $G(n,p)\not\rbarrow  H$. 
  Em particular, $\hat{p}_H \geq n^{-1/m_2(H)}$.
\end{teorema}


O Teorema \ref{teo:nen} foi estendido para ciclos \cite{barros2021anti} e para grafos completos \cite{kohayakawa2019anti} com mais de 5 vértices. 
O caso para o limiar do ciclo $C_4$ foi tratado por Mota em \cite{mota2017advances}.

\begin{teorema}    
[\cite{barros2021anti, kohayakawa2019anti, mota2017advances}]
\label{teo:cyclecomplete} 
Se $H$ é um ciclo ou um grafo completo de pelo menos $5$ vértices, 
então $\hat{p}_{H} \geq n^{-1/m_2(H)}$. 
Além disso, $\hat{p}_{C_4} = n^{-3/4}$ 
e $\hat{p}_{K_4} = n^{-7/15}$.
\end{teorema}

    O framework mencionado anteriormente de Nenadov,
Person, Škorić e Steger~\cite{NePeSkSt14} reduz o problemas de limitante inferior em grafos aleatórios para um problema determinístico em grafos com densidade máxima limitada. 
A saber, a propriedade de anti-Ramsey para um grafo fixo 2-balanceado $F$ possui como $0$-afirmação o limiar $p_F \geq n^{-1/m_2(F)}$ se todo grafo $G$ com $m(G) \leq m_2(F)$ é tal que $G \not\xrightarrow[]{rb} F$.
O resultado sobre ciclos do Teorema \ref{teo:cyclecomplete} é uma consequência do seguinte lema, sendo o próprio lema uma aplicação deste mesmo framework:
    
\begin{lema}[\cite{barros2021anti}]\label{lema:rbciclos}
         Sejam $\ell \geq 5$ um inteiro e $G$ um grafo tais que
         ${m(G) < (\ell-1)/(\ell-2) = m_2(C_\ell)}$. 
         Então, $G \not\xrightarrow[]{rb} C_\ell$.
\end{lema}
    
A essência da prova do Lema \ref{lema:rbciclos} está em usar uma estratégia de coloração própria nas arestas dos ciclos $C_\ell$ contidos no grafo $G$, de modo que toda cópia de ciclo $C_\ell$ tenha duas arestas coloridas com a mesma cor.
%
Para isso, os autores definem o que chamam de \emph{componente de ciclos} dentro de $G$ como uma sequência crescente de subgrafos $(H_1, \ldots, H_t)$ em $G$, em que a cada subgrafo $H_{i}$ existe uma cópia de $C_\ell$ que não está em $H_{i-1}$, para ${1 < i \leq t}$, com alguma interseção não-nula de arestas com $H_{i-1}$.
%
Eles observaram que tal sequência, com argumentos de densidade, impõe certas configurações restritas de como os novos ciclos aparecem entre os subgrafos, assim ao verificarem a ocorrência dessas configurações detalham alguns casos de como colorir arestas dessa sequência de subgrafos para evitar $C_\ell$ rainbow.
     Parte das ideias desse artigo são retomadas nas demonstrações descritas nas próximas seções.
 
 
   Para provar que o limiar de grafos completos do Teorema \ref{teo:cyclecomplete} de que $G(n,p) \rbarrow K_\ell$, com $\ell \geq 5$, seja igual a $\hat{p}_{K_\ell} = n^{-1/m_2(K_\ell)} = n^{- 2/(\ell+1)}$, os autores provaram o seguinte lema.
   % destacamos que o grafo $K_4$ tem um limiar diferente e é tratado à parte no mesmo artigo.
   
   \begin{lema}[\cite{kohayakawa2019anti}]\label{lema:rbcomplete}
         Sejam $\ell \geq 5$ um inteiro e $G$ um grafo tais que 
         $m(G) < (\ell+1)/2 = m_2(K_\ell)$.
         Então, $G \not\xrightarrow[]{rb} K_\ell$.
    \end{lema}
    
    Para demonstrar o Lema \ref{lema:rbcomplete}, é suficiente mostrar que todo grafo $G$ com $m(G) < (\ell+1)/2$  admite colorações próprias que evitam $K_\ell$ rainbow. 
    A primeira observação é que é suficiente considerar $G$ como um grafo onde toda aresta está contida em algum $K_\ell$, pois são elas que precisam ser examinadas para uma coloração sem cópias rainbow de $K_\ell$. 
    Os autores notam que a densidade limitada de $G$ implica na existência de ao menos um vértice $v$ de grau $\ell$, e, com esse vértice destacado apresentam uma prova indutiva na quantidade de vértices de $G$: consideram na hipótese indutiva ao subgrafo $G_v$ de $G$ obtido após remover $v$ e seus vizinhos que participam de algum $K_\ell$ em comum.
    
    Antes de dar a propriedade estrutural a ser preservada na indução, com argumentos usando a densidade limitada nos subgrafos de $G$, os autores analisam as possíveis configurações da vizinhança de $v$ quanto a formar cliques que se interceptam. 
    Dessas configurações, eles obtêm subcasos a tratar no passo indutivo que permite colorir $G$ evitando cópias rainbow de $K_\ell$.
    %% O parâmetro auxiliar $b(G)$ que contabiliza a proporção de arestas coloridas é relacionado a essas configurações e, com ele, a indução de $G_v$ para $G$ é construída notando que para diferentes valores em $b(G)$, cada $K_\ell$ do grafo será não-rainbow.
 
 
\section{Limiar para grafos bipartidos completos}\label{sec:limiar}

    Nesta seção começamos a discussão da prova para grafos bipartidos completos um resultado semelhante ao que foi comentado anteriormente sobre ciclos e grafos completos, a saber, a propriedade anti-Ramsey de grafos bipartidos completos em grafos aleatórios.
    Ou seja, dados os inteiros $\ell,r \geq 3$, pretendemos verificar que o limiar para a propriedade $G(n,p) \xrightarrow[]{rb} \K$ é dado por $n^{-1/m_2(\K)}$.

    Note que como toda coloração própria de arestas de uma estrela $K_{1,r}$ é obrigatoriamente rainbow, então
    o limiar da propriedade anti-Ramsey no caso de estrelas $K_{1,r}$ coincide com o limiar para que o grafo aleatório contenha uma estrela $K_{1,r}$.

    O caso de grafos bipartidos completos $K_{2,r}$, para $r\geq 1$ inteiro (o caso $r=2$, que corresponde ao $C_4$, é coberto pelo
    Teorema~\ref{teo:cyclecomplete}),
    nós provamos que $K_{2,3r-2} \rb K_{2,r}$, 
    o que implica que  a presença de um $K_{2,3r-2}$ no
    $G(n,p)$ guarante a existência de uma cópia rainbow de $K_{2,r}$ em qualquer coloração própria de arestas de~$G(n,p)$.  

     \begin{teorema}\label{teo:below} Seja $r\geq 1$ um inteiro, então
    $K_{2,3r-2} \rb K_{2,r}$.
     \end{teorema}
     \begin{proof} Nós provamos este resultado por indução em $r$. 
     O resultado claramente vale para $r=1$ pela coloração própria de $K_{2,1}$ ser rainbow. 
     Fixe $r\geq 2$ e suponha que $K_{2,3r-5} \rb K_{2,r-1}$. 
     Seja $G:=K_{2,3r-2}$ o grafo com partes $X$ e $Y$ tais que $|X|=2$ e $|Y|=3r-2$, e considere uma coloração própria qualquer de suas arestas.
     Então, como ela contém uma cópia de $K_{2,3r-5}$ e $K_{2,3r-5} \rb K_{2,r-1}$, então existe uma 
     cópia rainbow de $H_A:=K_{2,r-1}$ de partes $X$ e $A$ em $G$.
     Agora, considere  o subgrafo $H_B:=K_{2,2r-1}$ de $G$, com partes $X$ e $B$, onde $B$ é disjunto de $A$. 
     Uma vez que $H_A$ é colorido com $2r-2$ cores e $B$ possui
    $2r-1$ vértices, existe ao menos um vértice $b\in B$ tal que
    duas arestas de $H_B$ incidem em $b$ e não são usadas em $H_A$. 
    Portanto, existe uma cópia de $K_{2,r}$ obtida ao adicionar $b$ em $H_A$ que é rainbow, concluindo que  $K_{2,3r-2} \rb K_{2,r}$.
     \end{proof}
    
    Agora, como pelo Teorema \ref{teo:gnpcontem}  o limiar para
    $K_{2,3r-2}\subset G(n,p)$ é $n^{-1/m(K_{2,3r-2})}$ e sabendo das densidades que
    $m(K_{2,3r-2}) = (6r-4)/3r < (\ell r - 1)/(\ell+r-2) =
    m_2(K_{\ell,r})$, 
    temos que a presença de $K_{2,3r-2}$ no $G(n,p)$ força um limite superior menor do que
     $n^{-1/m_2(K_{2,r})}$, i.e., nós concluímos o seguinte Corolário:
     
    \begin{corolario}\label{cor:k2r}
        Seja $r\geq 2$ um inteiro, então $\hat{p}_{K_{2,r}} \ll n^{-1/m_2(K_{2,r})}$.
    \end{corolario}
    
    Nosso principal resultado (Teorema \ref{teo:mainrb}) mostra que, diferentemente do limiar de $K_{2,r}$,
    o limiar $\hat{p}_{\K}$, para $\ell,r \geq 3$ inteiros, é de fato da ordem 
    $n^{-1/m_2(\K)} = n^{-(\ell+r-2)/(\ell r-1)}$.
    A partir do Teorema \ref{teo:conditional}, temos o limite superior $\hat{p}_H \leq n^{-1/m_2(H)}$ para todo grafo $H$, e com isso, para determinarmos o limiar de $\hat{p}_{\K}$ é suficiente mostrar que 
    $\hat{p}_{\K}\geq n^{-(\ell+r-2)/(\ell r - 1)}$.

    \begin{teorema}\label{teo:mainrb} 
    Sejam $\ell,r \geq 3$ inteiros. 
    Então $\hat{p}_{K_{\ell,r}}\geq n^{-(\ell+r-2)/(\ell r - 1)}$.
    \end{teorema}

    %%yyyyy
    Uma vez que $\K$ é um grafo estritamente 2-balanceado, tal como fora utilizado em ciclos e grafos completos, pode-se pela aplicação do mesmo framework de Nenadov, Person, Škorić e Steger~\cite{NePeSkSt14} deduzir o Teorema \ref{teo:mainrb} a partir do seguinte resultado: %%que é provado na Seção ZZ.
    
    \begin{teorema}\label{lemma:main} 
        Sejam $\ell,r \geq 3$ inteiros e
        $G$ um grafo. 
        Se $m(G) < m_2(\K)$, então $G \not\rb \K$.
    \end{teorema}

    %%%yyyyy
    Na Seção \ref{sec:rbestrut} provamos alguns resultados estruturais que caracterizam como as cópias de $\K$ aparecem no grafo com densidade máxima limitada superiormente por $m_2(\K)$.
    Na Seção \ref{sec:mainres} provamos o Teorema \ref{lemma:main}.

\section{Resultados Estruturais}
\label{sec:rbestrut}

     A fim de provarmos o Teorema~\ref{lemma:main}, dado um grafo $G$ tal que
     $m(G) < m_2(\K)$, 
     precisamos obter uma coloração própria de arestas de $G$ sem cópias rainbow de~$\K$. 
     Para isto, precisamos analisar como se dão as interseções de arestas entre cópias de grafos bipartidos completos, dado que colorir tais arestas para evitarmos $\K$ rainbow pode exigir uma coloração engenhosa. 
     
     Começamos com a seguinte proposição cuja demonstração é obtida
     pela aplicação do chamado \textit{argumento de densidade}, este argumento examina as restrições em grafos geradas por uma densidade limitada superiormente e será muito utilizado nas provas desta seção.
    %xxxxx Como estamos interessados em grafos $G$ tais que  $m(G) < m_2(\K) = (\ell r - 1)/(\ell+r-2)$,  we state a simple observation that will be helpful when analyzing subgraphs of $G$.  
    Dado um grafo $G$, lembramos da notação $G[A,B]$ para representar a cópia de um grafo bipartido completo em $G$ com partes $A$ e $B$.
    Ao longo do texto, comumente escrevemos $3\leq \ell \leq r$ para ressaltarmos que o bipartido $\K$ pode possuir partes com tamanhos distintos. 
    A proposição a seguir descreve uma condição algébrica sobre o bipartido $K_{a,b}$ formado pela interseção de arestas entre duas cópias de $\K$ em $G$ cujos conjuntos de vértices de suas partes não se interceptam.

    %% caso a=b=0 a desigualdade continua valendo normalmente
    \begin{proposicao}
        \label{lemma:inter_a} 
    Sejam $3\leq \ell \leq r$ inteiros, $G$ um grafo, $G[A_1,B_1]$ e $G[A_2,B_2]$ cópias de $K_{\ell,r}$ em $G$. 
    Se $m(G) < m_2(\K)$ e
    $A_i \cap B_j = \emptyset$ 
    para $1\leq i,j\leq 2$, então o grafo bipartido completo
    $G[A_1\cap A_2, B_1\cap B_2]$ é uma cópia de $K_{a,b}$ tal que
      \begin{equation}
         \label{eq:pairs-ab} 
    	 \frac{2\ell r - ab}{2\ell + 2r - (a+b)} <
    \frac{\ell r - 1}{\ell + r - 2}.
      \end{equation}
      \end{proposicao}
    \begin{proof}
    Sejam $H_1:=G[A_1,B_1]$ e ${H_2:=G[A_2,B_2]}$ cópias de $\K$ em $G$.
    Da mesma forma, seja $H := G[a,b]$, o bipartido completo onde $a = A_1\cap
    A_2$ e $b = B_1\cap B_2$.  
    Uma vez que 
    $v(H_1 \cup H_2) = (\ell + r) +
    (\ell-a) + (r - b) = 2\ell + 2r - (a+b)$ 
    e $e(H_1 \cup H_2) = \ell r + (\ell-a)r + a(r-b)$, 
    da densidade $m(H) \leq m(G)$ e de $H\subset H_1\cup
    H_2$, temos que
      \begin{equation*} 
      \frac{\ell r + (\ell-a)r + a(r-b)}{2\ell + 2r -
    (a+b)} = \frac{2\ell r - ab}{2\ell + 2r - (a+b)} < m_2(\K) = \frac{\ell r -
    1}{\ell + r - 2},
      \end{equation*}
      o que conclui a prova.
    \end{proof}
    
    Faremos frequente aplicação, direta ou indiretamente, da Proposição~\ref{lemma:inter_a} em nossas provas.
     
     Vamos seguir uma estratégia próxima à encontrada no trabalho que estudou  a mesma propriedade anti-Ramsey em ciclos \cite{barros2021anti}, o qual analisou interseções de arestas entre ciclos no grafo alvo de coloração sob densidades limitadas.
     Dado um grafo $G$, seja $H = (H_1, \ldots, H_t)$ uma sequência crescente de subgrafos de $G$ tal que
    $H_1$ é uma cópia de $\K$ e, para cada $1 < i \leq t$, temos $H_{i-1} \subseteq H_i$ e existe alguma cópia de $\K$ em $H_{i}$ com interseção de arestas não-nula com $H_{i-1}$. 
    Ou seja, existe um $K \cong \K$ em $H_{i}$ tal que $K \not\subset H_{i-1}$ e $E(K) \cap E(H_{i-1}) \neq\varnothing$.
     Observe que a inclusão de uma cópia de $\K$ em $H_{i-1}$ pode até formar outras cópias de $\K$ em $H_{i}$ não presentes em $H_{i-1}$, sendo este um detalhe importante quando nos voltarmos para a coloração de $G$.
     
     Se a sequência $H$ for maximal, isto é, não pode mais ser estendida em $G$ quanto a cópias de $\K$, ela será chamada de $\K$-\emph{componente}.
     Queremos realizar uma coloração própria parcial da sequência $(H_1, \ldots, H_t)$ que evite cópias  rainbow de $\K$. 
     Será suficiente fornecer uma coloração para uma $\K$-componente arbitrária de $G$, dado que $G$ pode ser visto como um conjunto de $\K$-componentes disjuntas.
 
 Na Proposição \ref{lemma:inter_a}, dado um grafo $G$ com
 $m(G) < m_2(\K)$, nós descrevemos um bipartido como interseção que depende dos subgrafos obtidos nas cópias $G[A_1,B_1]$ e $G[A_2,B_2]$ de $K_{\ell,r}$ em $G$ sob a condição de $(A_1\cup A_2)\cap(B_1\cup B_2) = \emptyset$. %%%%%% reescrever parag
 A próxima proposição lida com o caso em que $A_1\cup A_2$ intersecta a $B_1\cup B_2$, e sua prova segue a mesma ideia da Proposition \ref{lemma:inter_a} com um argumento de densidade similar à desigualdade \ref{eq:pairs-ab}, porém apresenta uma solução única para esta, o que força uma configuração específica de intersecções de arestas.

%%% existe simetria entre l e r aqui? provavel que nao importe
    \begin{proposicao}\label{aff:casoP1}
       Sejam $3\leq \ell \leq r$ inteiros, 
      $G$ um grafo, e $K \cong G[A, B]$ uma cópia de $\K$ em $G$.
      Se $m(G) < m_2(\K)$ e existe em $G$ uma outra cópia
        $K' \cong \K $ que intercecta arestas de $K$ tal que 
        $(E_G[A] \cup E_G[B]) \cap K' \neq \varnothing$, 
        então $K'$ contém uma cópia de $K_{\ell-1,r-2}$ disjunta de $K$.
    \end{proposicao}
     \begin{proof}
         Seja $K$ uma cópia de $\K$ em $G$ com partes $A$ e $B$, queremos verificar se existe em $G$ o subgrafo $K'$ também cópia de $\K$ com partes $X$ e $Y$, onde pelo menos $A \cap X \neq \varnothing$ e $A \cap Y \neq \varnothing$.
         
         Sejam $X' = X\setminus A$,  $B' = B\cap Y$, $A' = A \cap Y$ e $Y' = Y \setminus (B\cup A')$, 
         onde
         $X'$ tem $x$ vértices, $A'$ tem $a$ vértices, $B'$ tem $b$ vértices e $Y'$ tem $y$ vértices (ver Figura \ref{fig:interP1}).


  \begin{figure}[htb] \centering \includegraphics[]{interP1.pdf}
   \caption{Interseção de arestas entre cópias de $\K$, $G[A,B]$ e $G[X,Y]$, usando arestas dentro da partição.}
   \label{fig:interP1}
 \end{figure}
         
         É evidente que $a \geq 1$, $b \geq 1$, e também que $a + b+y = r$ e $a + (\ell-x) \leq \ell$ são restrições da construção. 
         Reunindo essas informações e calculando a densidade como na Proposição~\ref{lemma:inter_a}, temos que 
         
         \[
            m(K \cup K') = \frac{\ell r + xr + (\ell-x)(b+y)}{\ell+r+x+y}
            < \frac{\ell r - 1}{\ell+r-2},
         \]
         
         de onde obtemos que a única solução inteira nessas condições é  $a=1$, $b=1$, $x = \ell-1$ e $y=r-2$. 
         De $X'$ e $Y'$ obtemos $K_{\ell-1, r-2}$ em $K'$ vértice disjunto de $K$.
         Uma parametrização semelhante usando arestas internas de ambos os conjuntos $A$ e $B$ não gera soluções.       
     \end{proof}
     
     Uma informação útil sobre a configuração dada na Proposição \ref{aff:casoP1} 
     % é de que ela limita a existência de uma interseção do tipo $K_{1,1}$ com outro $\K$. 
     % Logo, seria 
     é que tal configuração é densa o suficiente para ser uma $\K$-componente isolada do grafo $G$ se forem consideradas interseções semelhantes, sendo também restritiva, como veremos, para outros tipos de interseção. 
     Com isso, a coloração própria desta componente que evita cópias rainbow de $\K$ será trivial e poderemos nos voltar para outras interseções de arestas.
     
     Agora, podemos nos voltar para a intersecção entre duas cópias de $\K$ em $G$ que é da forma $K_{a,b}$ para alguns inteiros $a$ e $b$ (vide Proposição \ref{lemma:inter_a}). 
	 O grafo $G$ pode conter uma intersecção de múltiplas cópias de
    $\K$, e neste caso cada cópia de $\K$ seria um subgrafo de $K_{\ell+x,r+y}$ onde $x$ ou $y$ são inteiros não-negativos, e, como veremos, esta intersecção múltipla pode afetar a maneira como colorimos a $\K$-componente de $G$. %%%% explicar
    
    A partir da Proposição \ref{lemma:inter_a} também notamos que uma sequência de cópias de $\K$ que se interceptam em arestas possui menor densidade se as intersecções entre cada par de $\K$ da sequência forem da forma $K_{1,1}$. 
     Dada esta informação, computamos o limite inferior para a densidade de uma sequência de cópias de $\K$ em $G$, e obtivemos restrições sobre a estrutura das intersecções de arestas na $\K$-componente de $G$, como é apresentado no seguinte fato.
     
      \begin{fato}\label{fato:ciclos}
        Sejam $3\leq \ell \leq r$ inteiros e seja $G$ um grafo tal que $m(G) < m_2(\K)$.  
        Dada uma $\K$-componente $(H_1, \ldots, H_t)$ de $G$, 
        então o grafo bipartido $K \cong \K$ contido em $H_i$, mas não em $H_{i-1}$ é tal que ou $E_G(K) \cap E_G(H_j) = \varnothing$
        para todo $1 \leq j < i-1$, ou  $K \subseteq K_{\ell+s, r+w}$
        para algum valor de $0\leq s < \ell$, $0\leq w < r$, sendo $s$ e $w$ não ambos nulos. 
  \end{fato}
\begin{proof}
        Seja $G$ um grafo com densidade limitada $m(G) < m_2(\K)$. 
        As interseções de arestas entre somente duas cópias de $\K$ em $G$ da forma $K_{1,1}$ são as que apresentam menor densidade, e com isso, permitiriam o acréscimo máximo de arestas considerando qualquer sequência de subgrafos $(H_1, \ldots, H_t)$ que seja uma $\K$-componente. 
        
        Sem perda de generalidade, restringimos $G$ a sua maior $\K$-componente.       
        Defina $\mathcal{I}(G)$ como o \emph{grafo de interseção} de $G$ cujo conjunto de vértices é o conjunto de cópias de $\K$ em $G$ e cujas arestas correspondem a pares $\{K, K'\}$ tais que $K \neq K'$ e $E_G(K) \cap E_G(K') \neq \varnothing$.
         
        Suponha que existe um $s$-ciclo no grafo de interseção $\mathcal{I}(G)$ considerando só interseções da forma $K_{1,1}$, o que representa uma cadeia de cópias intersectantes de $\K$ em $G$.
        Teríamos em termos de densidade em $G$ dessa cadeia de interseções que
        \[
            \frac{s\ell r - s}{s(\ell+r) - 2s} =  \frac{\ell r - 1}{\ell+r - 2}
            \not < m_2(\K). 
        \]
        Logo, o grafo de interseção não contém ciclos considerando que as interseções são de um único $\K$ entre os subgrafos $H_i$ e $H_{i+1}$ da $\K$-componente, $1 \leq i < t$.
        
       Poderíamos tratar o caso de múltiplas cópias de $\K$ geradas por uma interseção entre $H_i$ e $H_{i+1}$ como um único vértice do grafo $\mathcal{I}(G)$, o que torna acíclico este último grafo pelo mesmo argumento de densidade. 
       Logo, qualquer cópia de $\K$  numa sequência $(H_1, \ldots, H_t)$ ou acaba de ser formada num determinado $H_i$ ou está contida em um $K_{\ell+s, r+w}$.
\end{proof}

Em outras palavras, o Fato \ref{fato:ciclos} afirma que cada bipartido $\K$ que surge em $H_i$ dentro da sequência de subgrafos $(H_1, \ldots, H_t)$ é tal que ou ele não intersecta arestas de termos anteriores a $H_{i-1}$ ou ele está contido em uma intersecção múltipla com a forma de $K_{\ell+s, r+w}$. 

      Nós usaremos as consequências da Proposição \ref{lemma:inter_a} e, com a análise de densidade de diferentes intersecções entre  os grafos bipartidos $\K$ na sequência de subgrafos $(H_1, \ldots, H_t)$, obtemos um resultado que caracteriza a configuração estrutural da cópia de $\K$ que aparece entre $H_{i-1}$ e $H_i$.
     Antes disso, precisamos definir os parâmetros $e_j$ e $v_j$, para cada  $1 < j \leq t$ inteiro:
     \begin{itemize}
         \item $e_j$ é o número de arestas em $E(H_{j}) \setminus E(H_{j-1})$;
         \item $v_j$ é o número de vértices em $V(H_{j}) \setminus V(H_{j-1})$. 
     \end{itemize}
     
     Note que $v_j \leq \ell+r-2$, pois por intersecção de arestas, o $\K$ entre $H_{j-1}$ e $H_{j}$ contém ao menos uma aresta de $H_{j-1}$, o que exige dois vértices em $H_{j-1}$. 
     Os $v_{j}$ vértices formam um $K_{\ell-a,r-b}$ em $H_{j}$, com $0 \leq a\leq \ell$ e $0\leq b\leq r$, onde $a,b$ são inteiros e ambos não assumem o valor máximo (mínimo) ao mesmo tempo, sendo que $v_{j} = \ell+r-a-b$.
 

  Para um conjunto de vértices $X = \{x_1, \ldots, x_\ell\}$, dados inteiros $1 \leq a \leq b \leq \ell$, denotamos por $X_a^b$ o subconjunto $\{x_a, \ldots, x_b\}$ de $X$. 

    Pretendemos exibir uma descrição matemática que consiga abranger as possibilidades de ocorrência das cópias do bipartido $\K$ numa dada $\K$-componente $(H_1, \ldots, H_t)$ de um grafo $G$.
    Seja $K = [X, Y]$ uma cópia de $\K$ em $H_i$, mas não em $H_{i-1}$, 
    onde $X = \{x_1, \ldots, x_\ell\}$ e $Y=\{y_1, \ldots, y_r\}$, para algum $2\leq i\leq t$
    e considere os inteiros $j\in [\ell]$, $k \in [r]$, $w \in \{0\}\cup [r]$ e $s \in\{0,1,2\}$.  
    Sejam $0 \leq p \leq \ell-j$, $0\leq q \leq r -k$ e $z = s+w$ inteiros, 
     definimos as seguintes \emph{configurações} para o bipartido $K$: 
     
     \begin{enumerate}
       \item[\hypertarget{HconfigA}{$(A_{jk})_{pq}$}] \label{configA} 
        os conjuntos de vértices $X$ e $Y$ são tais que $K$
    contém o subgrafo
    $K_{j,k} = \big[X_1^j, Y_1^k\big]$ em $H_{i-1}$, 
    e também formam um 
    $K_{p,q} = \big[X_{j+1}^{j+p}, Y_{k+1}^{k+q}\big]$
    com  vértices de $H_{i-1}$, onde $X$ e $Y$ são completados com vértices de
    $X_{j+p+1}^\ell$ 
    e
    $Y_{k+q+1}^r$ 
    em $H_i$, respectivamente;
                
       \item[\hypertarget{HconfigB}{$(B_{sw})$}]\label{configB} 
        Existem conjuntos de vértices $X$ e $Y$ em $H_{i-1}$, $Z_1$ e $Z_2$ em $H_i$ tais que
      $K_{s,w} = [Z_1, Z_2]$ e 
      $K_{\ell+s,r+w}= [X\cup Z_1, Y\cup Z_2]$ que contém~$K$.
      \footnote[1]{Um caso especial se dá quando $z = 0$ e 
     ou $Z_1$ ou $Z_2$ é um conjunto unitário, com isso, $V(H_i) = V(H_{i-1})$.}
       \end{enumerate}
  
  %%% multiple figures: https://tex.stackexchange.com/questions/271518/multiple-panel-figure-with-figures-side-by-side 
  \begin{figure}[htb] \centering \includegraphics[width=0.94\textwidth]{configA.pdf}
   \caption{Configuração do tipo $A$.}
   \label{fig:configA}
 \end{figure}

    Nós também dizemos que $H_i$ é de uma das configurações, $(A_{jk})_{pq}$ ou $(B_{sw})$, 
    quando $H_i = H_{i-1} \cup K$ e satisfaz as condições acima. 
	De maneira geral, dizemos que $K$ é do \textit{tipo A} se $K$ é da configuração $(A_{jk})_{pq}$, analogamente,
		$K$ é do \textit{tipo B} se $K$ é de uma configuração $(B_{sw})$ -- ver Figuras \ref{fig:configA} e \ref{fig:configB}.
    As configurações de $K$ que são descritas pela Proposição~\ref{aff:casoP1} serão denominadas do \textit{tipo C} (Figura \ref{fig:configC}).
    
    O resultado a seguir conjuga as definições e propriedades anteriores da seção mostrando que as cópias de $\K$ num grafo $G$ com densidade limitada são de configurações que se restringem aos tipos $A, B$ ou $C$. 
    Com isso, obtemos uma caracterização estrutural dos grafos bipartidos para numa etapa posterior realizarmos coloração de $G$ que evita $\K$ rainbow.
  
%%multiple figs:  https://tex.stackexchange.com/questions/119905/insert-multiple-figures-in-latex

   %Configurações $(A_{jk})_{pq}$ e $(B_{sw})$ de um $\K$ contido em $H_i$, mas não em $H_{i-1}$.
  
 \begin{figure}[htb] \centering \includegraphics[width=0.95\textwidth]{configB.pdf}
   \caption{Configuração do tipo $B$.}
   \label{fig:configB}
 \end{figure}

   \begin{figure}[htb] \centering \includegraphics[width=0.95\textwidth]{configC.pdf}
   \caption{Configuração do tipo $C$.}
   \label{fig:configC}
 \end{figure}

 \begin{lema}\label{lema:config}
        Sejam $3\leq \ell \leq r$ inteiros, e $G$ um grafo tal que $m(G) < m_2(\K)$, 
     e $H = (H_1, \ldots, H_t)$ uma $\K$-componente de $G$.  
    Então,  vale para cada $2\leq i \leq t$: 
    se $K = [X, Y]$ é uma cópia de $\K$ em
    $H_i$ e não em $H_{i-1}$, 
    então $K$ é do tipo A, B ou C.   
    \end{lema}
\begin{proof}
        Dados $3\leq \ell \leq r$ inteiros,
        sejam $G$ um grafo tal que $m(G) < m_2(\K)$ e
        $H = (H_1, \ldots, H_t)$ uma $\K$-componente de $G$.
        Vamos provar o resultado sobre os tipos de configuração que $K$, uma cópia de $\K$, apresenta ao ser formado entre $H_{i-1}$ e $H_i$ arbitrários para cada número possível de vértices $v_i$, i.e., 
        para qualquer $0 \leq v_i \leq \ell + r-2$. 
        
        O caso mais simples é o bipartido $K$ formar uma configuração do tipo C que só ocorre se 
     $v_i = (\ell-1)+(r-2) = \ell+r-3$ 
     e se $K$ seguir as condições da interseção apresentada pela Proposição \ref{aff:casoP1}.        
       
    Para avaliar os tipos A e B, primeiro construímos conjuntos auxiliares que descrevam as intersecções de arestas de grafos bipartidos em $G$.   
    Dados os inteiros positivos $\ell$ e $r$, denote por
    $\mathcal{S}_{\ell,r}$ o conjunto de pares de inteiros $\{a,b\}$ tal que a desigualdade~\ref{eq:pairs-ab} vale.  
    Para descrever $\Slr$, antes definimos os seguintes conjuntos de pares inteiros
    \begin{align}\label{eq:conjS} 
        \mathcal{S} &:= \left\{\{1,1\},\{1,2\},\{1,3\},\{2,2\}\right\},
    \\
    	\mathcal{S}_\ell &:=
    \left\{\mathcal{S},\{\ell-1,\ell-1\},\{\ell-2,\ell\},\{\ell-1,\ell\}\right\}.
    \end{align} 
        
    Também definimos os conjuntos de inteiros consecutivos 
    \begin{equation}
        \begin{aligned}[b]
                \beta_1 &:=
                \left\{\left\lceil \frac{\ell^2r - 4\ell r + \ell+2r}{(\ell-1)^2}\right\rceil,\dots, r-1\right\}   
                \\
                \beta_2 &:=
                \left\{\left\lceil \frac{-\ell^2r +5\ell r - \ell-2r-1}{r-\ell^2+3\ell-3}\right\rceil,\ldots, r\right\}.
            \end{aligned} 
        \label{eq:conjBeta}
    \end{equation}
    
    
    Uma verificação algébrica da desigualdade da Proposição~\ref{lemma:inter_a} nos dá os elementos de $\mathcal{S}_{\ell,r}$ e estes são apresentados a seguir (os conjuntos $\mathcal{S}_{\ell,r}$ para casos menores, quando $3 \leq \ell, r < 7$, são descritos no Apêndice \ref{cap:app}):
    
      \begin{enumerate}[(i)]
      \item \label{it:Si} se $r=\ell = 3$ ou $r=\ell \geq 7$, então
    $\mathcal{S}_{\ell,r} = \mathcal{S}_\ell$;
      \item  se $r=\ell = 4$, então $\mathcal{S}_{\ell,r} =
    \mathcal{S}_\ell \cup\big\{\{2,3\}\big\}$;
      \item se $r=\ell = 5$, então $\mathcal{S}_{\ell,r} =
    \mathcal{S}_\ell \cup\big\{\{2,3\},\{3,3\},\{3,4\}\big\}$;
      \item se $r=\ell = 6$, então $\mathcal{S}_{\ell,r} =
    \mathcal{S}_\ell \cup\big\{\{2,3\},\{3,3\},\{4,5\}\big\}$;
      \item \label{it:Sv} if $7 \leq \ell < r \text{ e } r > \ell^2-3\ell+3$,
    então $\mathcal{S}_{\ell,r} = \mathcal{S} \cup \{\{a,b\}: a=\ell\text{
    e }b\in \beta_1\}$;
       \item \label{it:Svi} se $7 \leq \ell < r \text{ e } r \leq
    \ell^2-3\ell+3$, então $\mathcal{S}_{\ell,r} = \mathcal{S} \cup
    \{\{a,b\}: a=\ell\text{ e }b\in \beta_1;\text{ ou }a=\ell-1\text{
    e }b\in \beta_2\}$.
      \end{enumerate}

     \begin{figure}[htb] \centering 
      \includegraphics[]{Kll-inter.pdf}
       \caption{Grafos bipartidos completos $K_{a,b}$ obtidos na interseção quando $\ell=r$ seguindo os pares $\{a,b\} \in\mathcal{S}_\ell$.}
       \label{fig:configKLL}
     \end{figure}

    Agora, prosseguimos dividindo os casos por $v_i$ e também se $K$ forma mais de uma cópia de $\K$ em $H_i$. 
    Os valores que $v_i$ pode assumir depende diretamente dos conjuntos $\Slr$, pois $K$ forma interseções de arestas entre $H_{i-1}$ e $H_i$ com quantidades delimitadas por $\Slr$.
    
    \medskip \textbf{Caso 1.}
	\emph{Se $3 \leq v_i \leq \ell+r-2$ e somente uma cópia de $\K$ é formada entre $H_{i-1}$ e $H_i$}.
       
    Seja $v_i= \ell+r-a-b$, onde $a,b$ são inteiros positivos.  
    Os $v_i$ vértices dessa única cópia $K \cong \K= [X,Y]$ formam um bipartido $K_{\ell-a,r-b}$ em $H_i$; 
    a $\K$-componente $H$ é composta pelas intersecções de arestas de $K$ com $H_{i-1}$, 
    o que pela Proposição \ref{lemma:inter_a}, elas formam bipartidos completos $K_{j,k}$,
    onde $\{j,k\} \in \{\{x,y\} \in \Slr: x < \ell \text{ e } y < r\}$, pois se $j=\ell$ ou $k=r$ teríamos mais de uma cópia de $\K$ formada na sequência $H$.  
   
   Além disso, completamos a partição $[X,Y]$ tomando outros $p$ vértices em $H_{i-1}$ para $X$ e $q$ vértices em $H_{i-1}$ para $Y$, estes vértices são tais que um bipartido $K_{p,q}$ é obtido satisfazendo às condições de  $j+p = a$ e $k+q = b$. 
   
   Suponha que já haviam arestas em $K_{p,q}$  para algum $H_{i'}$ na sequência de subgrafos~$H$ com $i' < i-1$.
   Logo, o novo $K$ adicionado teria interseção de arestas com algum outro bipartido completo $K'$, além do bipartido $K''$ que contém $K_{jk}$ em $H_{i-1}$.
   Assim, considerando o subgrafo $H_{i}$,
   %teríamos que existe um caminho entre $K'$ e $K''$, e 
   temos que $K$ 
   intersecta arestas de algum $H_{i'}$ para $i'<i-1$, o que contradiz o Fato \ref{fato:ciclos}.
  
     Com isso, as arestas de $K_{p,q}$ são adicionadas por $H_i$ na $\K$-componente e, nós obtemos que $K$ é da configuração %~\hyperref[configA]{$(A_{jk})_{pq}$},
     \hyperlink{HconfigA}{$(A_{jk})_{pq}$}
     ou seja, $K$ é do tipo A.  
    Esta configuração formou um único $\K$ entre $H_{i-1}$ e $H_i$, agora consideramos casos onde mais de uma cópia de $\K$ venha a aparecer em $H_i$.

    Sejam $b_1, b_2$ inteiros positivos tais que $b_1 \in \beta_1$ ou $b_2 \in \beta_2$, onde os conjuntos $\beta_1$ e $\beta_2$ foram definidos em (\ref{eq:conjBeta}).
    
     \medskip \textbf{Caso 2.} 
   \emph{Se $v_i \in \left[1, \max\{r - b_1, r+1-b_2\}\right]$ e existem múltiplas cópias de $\K$ entre $H_{i-1}$ e $H_i$}.

    Na sequência $H$, o acréscimo de $K$ em $H_{i-1}$ forma múltiplas cópias novas de $\K$ em $H_i$, o que pela maneira que se dá a interseção de arestas vista na Proposição \ref{lemma:inter_a} 
    %%e pelos conjuntos $\Slr$ 
    teríamos que 
    $K \subset K_{\ell+s, r+w}$, onde $v_i = s+w$ com $s,w \geq 0$ inteiros. 
    Resta-nos restringir os valores de $s$ e $w$ para verificarmos que estes coincidem com a descrição da configuração do tipo B.
    %% Logo, $K$ teria a configuração~\hyperlink{HconfigB}{$(B_{s,w})$}

    Considere os conjuntos $\Slr$ especialmente nos casos (\ref{it:Si}) e (\ref{it:Sv})-(\ref{it:Svi}) para notar que $K_{\ell+s, r+w}$ implica que $s \in \{0,1,2\}$.
    Seja $Z=Z_1\cup Z_2$ o conjunto com $v_i$ vértices tal que 
   $K_{sw}=[Z_1,Z_2]$ é formado em $H_i$. 
   A configuração~\hyperlink{HconfigB}{$(B_{1,w})$} é possível 
   se existe exatamente uma aresta em $Z$, digamos que $uv$, e então teríamos que
    $Z_1 = \{u\}$ e $v \in Z_2$, onde $w = r-b_2$.
        
    Outro subcaso ocorre se não há arestas em $Z$, assim $Z_1 =
    \varnothing$ e obtemos a configuração~\hyperlink{HconfigB}{$(B_{0,w})$}, 
    onde $w = r-b_1$.
    Um último subcaso ocorre quando $\ell = r$ e permite a
configuração~\hyperlink{HconfigB}{$(B_{2,0})$}. 
    
    Com isso, neste caso obtivemos que o bipartido $K$ é do tipo B.
   
 \medskip \textbf{Caso 3.} 
	 \emph{Se $v_i = 0$ e existem uma ou mais cópias de $\K$ entre $H_{i-1}$ e $H_i$}. 
	  
	 Neste caso temos que toda nova aresta de $H_i$ se liga a vértices pré-existentes de $H_{i-1}$. 
      Se considerarmos somente a adição de arestas em uma única cópia de $K' \cong \K$ em $H_{i-1}$, veremos que não é possível criar outro bipartido $K$, pois pelo argumento de densidade temos que a adição de $e_i$ arestas em $K'$ nos dá
      
    \[ 
		\frac{\ell r + e_i}{\ell+r} < \frac{\ell r-1}{r+\ell-2},
    \]
	e isto implica sob as condições de $3 \leq \ell \leq r$ que 
	$e_i \leq (2\ell r -\ell-r)/(\ell+r-2) < \ell+r-2$, 
	e esse limitante de $e_i$ o torna insuficiente na quantidade de arestas que se possa, por exemplo, inverter dois vértices da partição de $K'$ que resulte em um novo $\K$.
    Com isso, para $v_i=0$, resta-nos analisar intersecções de mais de uma cópia de $\K$ em $H_{i-1}$ para formar $K$.
    
    Para as intersecções possíveis entre dois bipartidos completos $K'$ e $K''$, descritas na Proposição \ref{lemma:inter_a} que formam um $K_{a,b}$, com $\{a,b\} \in \Slr$, realizamos a  análise da seguinte desigualdade de densidades:

    \begin{equation}\label{eq:inter}
        \frac{2\ell r - e(K'\cap K'') + e_i}{2\ell+2r - v(K' \cap K'')} < \frac{\ell r- 1}{\ell+r-2}.
    \end{equation}
	    
     Em particular, listamos as configurações possíveis para as interseções seguindo os conjuntos $\Slr$ que resolvem (\ref{eq:inter}):
     \begin{itemize}
         \item as intersecções $K_{1,3}$, $K_{2,2}$ são  justas, isto é, são subgrafos de $H_{i-1}$ cujo argumento de densidade em (\ref{eq:inter}) requer $e_i=0$, i.e., não permitem acréscimo de arestas.
         
         \item para $K_{\ell-2,r}$, temos $e_i < \ell$ que também não forma outro $\K$.
         
        \item se a intersecção for $K_{1,1}$, para todos os valores de $\ell$ e $r$ temos que $e_i = \ell-1$, e com isso podemos gerar um novo $\K$ de $H_{i}$, basta escolher um vértice de fora da interseção e adicionar a ele $e_i$ arestas de modo que forme $K_{\ell+1,r}$, logo, de configuração~\hyperlink{HconfigB}{$(B_{1,0})$}; analogamente, temos se $\ell \in
        \{r-1, r\}$ para $e_i = r-1$, que uma configuração possível é~\hyperlink{HconfigB}{$(B_{0,1})$}.
        
        \item  para a interseção $K_{\ell-1, r-1}$ temos uma única solução $e_i=1$, e podemos dela obter $K_{\ell+1, r}$ ou $K_{\ell,r+1}$, logo, de configurações~\hyperlink{HconfigB}{$(B_{1,0})$}
    e~\hyperlink{HconfigB}{$(B_{0,1})$}, respectivamente.
    
        \item A intersecção $K_{1,2}$ admite $e_i = \ell-1$, se $r > \ell^2 - 3\ell+3$ e isso permite a configuração~\hyperlink{HconfigB}{$(B_{1,0})$}.
        
        \item  As outras intersecções da forma $K_{\ell, r-b_1}$ e $K_{\ell-1,b_2}$ -- as quais o conjunto $\Slr$ depende de $\beta_1$ ou de $\beta_2$--,  de (\ref{eq:inter}) seguem as restrições $e_i < r$ e $e_i < \ell$, respectivamente; no entanto,
    	para com elas obter outro $\K$ em $H_i$ precisaríamos adicionar pelo menos
    	$r$ e $\ell-1$ arestas, respectivamente, a um vértice. Logo, não formam cópia de $\K$. 
     \end{itemize}
      
   Agora, examinamos o subcaso em que o acréscimo de arestas se dá entre duas cópias disjuntas de $\K$ em $H_{i-1}$, digamos que $K'$ e $K''$, para formar um outro bipartido $K\cong\K$.
    Como $K'$ e $K''$ estão na mesma $\K$-componente e são disjuntos, $K'$ ou $K''$ deve estar contido em um subgrafo $H_{i'}$ da sequência dessa componente para algum $i'<i-1$.
    Com isso, teríamos que $K$ em $H_i$ intercepta arestas de $H_{i'}$, o que contraria o Fato \ref{fato:ciclos}, logo, quaisquer arestas novas entre duas cópias disjuntas de $\K$ em $H_{i-1}$ não podem formar um novo bipartido $K$.

\end{proof}
  
  A Figura \ref{fig:configKLL} ilustra as configurações de intersecção de arestas entre $H_i$ e $H_{i+1}$ dadas pela Proposição~\ref{lemma:inter_a} quando $\ell=r$.

\section{Limiar em $n^{-1/m_2(K_{\ell,r})}$}
\label{sec:mainres}

Nesta seção provamos o Teorema~\ref{lemma:main}, ele segue da construção de uma coloração própria de arestas de um grafo $G$ arbitrário em que $m(G) < m_2(\K)$, e mostramos que esta coloração  não contém cópias rainbow de~$\K$, e com isso, $G \not\rbarrow \K$.

Na seção anterior, obtivemos propriedades estruturais de como os grafos bipartidos completos são formados em $G$, especialmente o Lema \ref{lema:config} que os sintetiza em três tipos de configurações: A, B e C.
Nosso resultado principal é semelhante aos Lemas \ref{lema:rbciclos} e \ref{lema:rbcomplete}, pois exibe colorações sem determinadas cópias rainbow. 
Uma parte importante de nossa prova aqui é verificar que as configurações do tipo B 
-- que geram múltiplos $\K$ entre cada subgrafo de uma $\K$-componente de $G$ -- 
são menos presentes por conta da densidade elevada que possuem, e, aparecem no máximo duas vezes seguidas. 
  Isso torna possível fazer uma coloração própria parcial de $G$ que evita cópias de $\K$ rainbow, o que nos permite concluir o resultado de que $G \not\rbarrow \K$.
  A prova segue abaixo:  
  
    \begin{proof}[Demonstração do Teorema~\ref{lemma:main}]
      Sejam $3 \leq \ell \leq r$ inteiros e $G$ um grafo tal que $m(G) < (\ell r - 1)/(\ell+r-2)$.
    
        Tome um bipartido $\K = [X,Y]$ arbitrário em $G$ para ser $H_1$ e atribua uma cor distinta $c_j$ para $\ell-3$ arestas paralelas entre $X$ e $Y$, a cada $\ell-3$ vértices de $Y$ em $H_1$, onde
        $1 \leq j \leq \lfloor \frac{r}{\ell-3}\rfloor$. 
        Com isso, a cópia de $\K$ em $H_1$ não é rainbow.
        Excepcionalmente nesta prova, se $\ell \in\{3,4\}$, colorimos as arestas paralelas de mesma cor sempre como um único par.

        Seja $H = H_t$, com $t \geq 2$, uma $\K$-componente de $G$ obtida por uma sequência de subgrafos $(H_1, \ldots, H_t)$ construída pela interseção de bipartidos completos.

         Vamos considerar as três configurações dadas pelo Lema \ref{lema:config} que ocorrem nesta sequência $(H_1, \ldots, H_t)$.
         %
         Para cada $1 < i \leq t$, existe pelo menos um bipartido $\K$ em $H_{i}$ que não está em $H_{i-1}$.
         Nosso objetivo é atribuir cores para as arestas em $E(H_{i})$ evitando cópias rainbow de $\K$ para qualquer tipo de configuração, provando isto, consequentemente, teremos que $G \not\rbarrow\K$, pois o resultado sobre $\K$-componentes cobre todas as cópias de $\K$ em $G$.
         Dito isso, dividimos a coloração de $E(H_i)$ em três casos seguindo os tipos A, B e C de configuração de $H_i$:

         \medskip \textbf{Caso A.} Se $H_i$ é do tipo A, ou seja, apresenta configuração~ \hyperlink{HconfigA}{$(A_{jk})_{pq}$}, onde $j < \ell$ e $k < r$.
         
         %%xxxxx if $\ell \geq 7$, 
        Neste caso é adicionado um único $\K = [X,Y]$ em $H_i$.     

        Como $\{j,k\} \in \{\{x,y\} \in \Slr: x < \ell \text{ e } y < r\}$, então existem em $H_i \setminus H_{i-1}$ pelo menos dois vértices, digamos que $u\in X$ e $v\in Y$, com arestas novas de modo que 
        %%%% a cada par $(u,v)$, 
        usamos $\ell-3$ cores novas para as arestas de $u$ para os vértices de $Y\setminus\{v\}$ e repetimos tais cores nas arestas de $v$ para $X\setminus\{u\}$; é possível realizar esse processo de coloração exaustivamente com novas $\ell-3$ cores para cada par de vértices $u\in X$ e $v\in Y$ que sejam de $H_i\setminus H_{i-1}$ e que não tiveram arestas coloridas.
        Assim, temos arestas de mesma cor que são paralelas em uma cópia de $\K$ para não ser rainbow.
        A Figura \ref{fig:configAcor} ilustra a coloração para esta configuração.

 \begin{figure}[htb] \centering \includegraphics[width=0.95\textwidth]{tipoAcolor.pdf}
   \caption{Esquema de coloração para configurações do tipo $A$.}
   \label{fig:configAcor}
 \end{figure}
        %% em $X$  (por conta dos pares em $\mathcal{S}$, vide (\ref{eq:conjS})) que podem ser coloridas com o mesmo processo usado em $H_1$.        
        %% Se $j = \ell-1$ e $k=r-1$, então podemos atribuir novas cores a cada $\ell-3$ vértices de $Y$ incidentes em $X$.

        \medskip \textbf{Caso B.} Se $H_i$ é do tipo B, ou seja, apresenta configuração~\hyperlink{HconfigB}{$(B_{sw})$}, onde $w \in \{0\}\cup [r]$ e $s \in\{0,1,2\}$.
        
        Aqui, vamos reutilizar alguma cor presente em $H_{i-1}$, pois neste caso a configuração do tipo B gera múltiplos $\K$ ao formar um $K_{\ell+s,r+w}$ em $H_i$, este último bipartido que intersecta cópias de $\K$ com arestas coloridas em passos anteriores da $\K$-componente.
        
      Considere a configuração~\hyperlink{HconfigB}{$(B_{0,w})$} em
      $H_i$ com $w > 2$, então existem $w$ novos vértices de um conjunto $Z$ tal que o bipartido formado é $K \cong K_{\ell,r+w} =[X, Y \cup Z]$, sendo que $[X,Y]$ já possui arestas coloridas em passos anteriores. %%onde $2 < w \leq r$ 
       Faremos um processo de coloração semelhante ao visto anteriormente, atribuímos uma nova cor a cada $\ell-3$ arestas paralelas com pontas entre $X$ e $Z$. 
      É importante notar que tomando $r$ vértices quaisquer de $Y \cup Z$, tem-se que algum par de vértices envia arestas de mesma cor com pontas em $X$.
      Com isso, cada cópia dos bipartidos $\K$ formados em $H_i$ não é rainbow para~\hyperlink{HconfigB}{$(B_{0,w})$}, onde $w >2$. 
      A Figura \ref{fig:configBcor} esboça esta coloração de~\hyperlink{HconfigB}{$(B_{0,w})$}.
      Note que esse procedimento de coloração pode ser facilmente usado para configurações~\hyperlink{HconfigB}{$(B_{s,w})$} com $s>0$ e $w>2$, onde adaptamos para o bipartido formado $K_{\ell+s,r+w} =[X\cup Z_1, Y \cup Z_2]$ que colorimos as arestas paralelas entre $X$ e $Z_2$ e entre $Y$ e $Z_1$.


 \begin{figure}[htb] \centering \includegraphics[width=0.96\textwidth]{tipoBcolor.pdf}
   \caption{Esquema de coloração para configurações do tipo $B$.}
   \label{fig:configBcor}
 \end{figure}
      
	  No caso da configuração~\hyperlink{HconfigB}{$(B_{0,2})$},
   temos  o bipartido $K_{\ell, r+2} = [X, Y \cup Z]$, onde $Z = \{u,v\}$ possui os vértices novos de $H_i$.
   Vamos reutilizar uma cor $c$ de $H_{i-1}$ nas arestas de $u$ e $v$ que incidem em $X$ e que não receberam a cor $c$ ainda.
   Além desta cor $c$, colorimos pares de arestas paralelas entre $Z$ e $X$ com outras $\ell-2$ cores novas (ver Figura \ref{fig:configBcor}).   
        A coloração é análoga para a configuração~\hyperlink{HconfigB}{$(B_{1,1})$}. 
        A viabilidade de reutilizarmos cores de $H_{i-1}$ é discutida no final da demonstração.

%        Quando $\ell = r$, a configuração~\hyperlink{HconfigB}{$(B_{2,0})$} representa um       $K_{\ell, \ell+2} = [X, Y \cup Z]$, onde $Z = \{u,v\}$, e $[X,Y]$ já possui $\ell-3$ arestas coloridas com uma certa cor $c$. 
    % Então é suficiente repetir a mesma cor $c$ em duas arestas entre $u,v$ e dois vértices de $X$ que não receberam a cor $c$ ainda.
        
       A configuração~\hyperlink{HconfigB}{$(B_{1,0})$} vista na prova do Lema \ref{lema:config} possui três subcasos que a geram: 
       as intersecções da forma $K_{1,1}$, $K_{1,2}$ e $K_{\ell-1,r-1}$, respectivamente. 
       Nos primeiros dois subcasos, existe um vértice de $K \cong K_{\ell+1,r}$, digamos que $v$,
        fora da interseção de $K_{1,1}$ ou $K_{1,2}$ em que podemos reusar uma
        cor $c$ (pois cada cor é usada em no máximo $\ell-3$ arestas paralelas) atribuindo a cor $c$ a uma aresta incidente em $v$, e isso faz com que toda cópia $\K$ em $K$ não seja rainbow -- a Figura \ref{fig:configBcor} também exibe a coloração deste subcaso.
                
        No terceiro subcaso, temos o acréscimo de uma única aresta e isto pode nos forçar a colorir alguma aresta já presente em $H_{i-1}$, então precisamos verificar que existem arestas não coloridas em $H_{i-1}$. 
        Dito isso, o subgrafo $K \cong K_{\ell+1,r}$ tem uma aresta, digamos $vw$, adicionada em $H_i$, e pelo menos $\ell-3$ arestas pré-coloridas na cor $c$;
        se tais arestas não contém $v$ e nem $w$, então colorimos $vw$ com $c$, caso contrário, atribuímos uma cor nova $c'$ a $vw$ e para outra aresta paralela a $vw$ que ainda não fora pintada em $K$.  
        Uma coloração análoga pode ser dada para a configuração~\hyperlink{HconfigB}{$(B_{0,1})$}.
        
        Para concluir, precisamos verificar que o método de coloração dado anteriormente é válido, isto é, de que existem arestas livres de cor quando for necessário reutilizar alguma cor neste processo.
        Veremos que duas configurações~\hyperlink{HconfigB}{$(B_{sw})$} não ocorrem de maneira sucessiva na sequência de subgrafos  de $G$ que é uma $\K$-componente, pois isso concentraria arestas nos mesmos vértices afetando sua densidade. %%, exceto se forem dois $(B_z)_1$ que equivalem a $(B_z)_2$ quando $z \in \{1,2\}$.  
       %% O caso~\hyperlink{HconfigB}{$(B_{2,0})$} não pode ser sucessivo no mesmo subconjunto de vértices, pois geraria densidades proibitivas com relação a $m_2(\K)$.
       %
        %% de subgrafos tais como $K_{\ell+2,\ell+2}, K_{\ell, \ell+4}, K_{\ell+1,\ell+1}$.
        Analisamos a configuração~\hyperlink{HconfigB}{$(B_{sw})$} de menor densidade que é~\hyperlink{HconfigB}{$(B_{1,0})$}.
        Para $K \cong K_{\ell+1,r}$ advindo de um~\hyperlink{HconfigB}{$(B_{1,0})$}, com interseção $K_{1,1}$, teríamos uma densidade em que 
        
        \[
            \frac{2\ell r+\ell-2+e_i}{2\ell+2r-2} < \frac{\ell
r-1}{r+\ell-2},
        \]
        \\
        e isto permite $e_i = \ell-1$ para $r > 2\ell^2-5\ell+4$,
        de modo que uma nova cópia $\K$ de configuração~\hyperlink{HconfigB}{$(B_{1,0})$} seria possível como subgrafo da sequência da $\K$-componente -- bastaria inverter um par de vértices na bipartição para isto.
         Podemos colorir seguindo a mesma ideia se a configuração anterior for~\hyperlink{HconfigB}{$(B_{1,0})$}, pois nesta intersecção possui duas cópias de $\K$ já coloridas com pelo menos duas cores.
        %
        Enquanto que a configuração~\hyperlink{HconfigB}{$(B_{1,0})$}
que advém das intersecções $K_{1,2}$ e $K_{\ell-1,r-1}$ não podem formar outra cópia de $\K$.  

 
        Fora isso, as outras configurações que exigiriam colorir arestas pré-existentes em $H_{i-1}$ não podem se suceder na sequência da $\K$-componente de $G$ por apresentarem uma densidade maior do que~\hyperlink{HconfigB}{$(B_{1,0})$}.
        Com isso, na maioria das configurações para $H_i$ apenas colorimos dentre as $e_i$ arestas presentes a partir de $H_i$.

        \medskip \textbf{Caso C.} Se $H_i$ é do tipo C, ou seja, apresenta a configuração dada na Proposição~\ref{aff:casoP1}.

        Como o bipartido adicionado em $H_i$ contém um $K_{\ell-1,r-2}=[X,Y]$, podemos utilizar uma coloração de arestas semelhante ao caso A.
        
     \end{proof}
 
  Como discutido ao final da Seção \ref{sec:limiar}, a demonstração do Teorema \ref{lemma:main} junto ao framework de Nenadov, Person, Škorić e Steger~\cite{NePeSkSt14} 
  implicam o resultado sobre o limiar da propriedade anti-Ramsey para $G(n,p) \rbarrow \K$ ser de $n^{-1/m_2(\K)}$ (Teorema \ref{teo:mainrb}).

    
\section{Problemas em aberto}
\label{subsec:bipartido-prob}
Ao longo deste capítulo, estávamos interessados em investigar quais classes de grafos $H$ apresentam função limiar para a propriedade $G(n,p)\rbarrow H$  dada por $n^{-1/m_2(H)}$, 
e verificamos isso para quase todos os grafos bipartidos completos.
Ademais, ainda há bastante campo de pesquisa com relação às funções limiares para outras classes de grafos $H$. 
Assim, a partir do que estudamos e do que se sabe de ciclos e grafos completos, o próximo passo é confirmar ou contrapor as seguintes conjecturas. % que podem ter respostas negativas a partir de contraexemplos.

%%%xxx \begin{conjectura}\label{conj:antiKll}      A propriedade $G(n,p)\rbarrow \K$ tem função limiar em  $p(n) =  n^{-1/m_2(\K)}$ para todo inteiro $\ell \geq 3$. \end{conjectura}


\begin{conjectura}\label{conj:antibipartidos}
     A propriedade $G(n,p)\rbarrow H$ tem função limiar em  $p(n) =  n^{-1/m_2(H)}$ quando $H$ é um grafo bipartido (não necessariamente completo) de pelo menos 5 vértices.
\end{conjectura}

\begin{conjectura}\label{conj:antiregulares}
     Seja $k \geq 3$ inteiro, a propriedade $G(n,p)\rbarrow H$ tem função limiar em  $p(n) =  n^{-1/m_2(H)}$ quando $H$ é um grafo $k$-regular.
\end{conjectura}

É importante ressaltar que já há trabalhos  que investigam a propriedade anti-Ramsey de cliques quando o grafo a ser colorido segue outro modelo de distribuição probabilística \cite{aigner2022large, aigner2022small}, o que mostra que há mais variantes do problema com oportunidades de pesquisa.



\chapter[Considerações finais]{Considerações finais}
\label{cap:conclusao}

A temática de coloração tipo anti-Ramsey tem-se mostrado frutífera em  problemas e conjecturas
com resultados relevantes tanto em grafos determinísticos, quanto em grafos aleatórios. 

Neste trabalho, primeiro abordamos colorações próprias de arestas do grafo completo que evitam repetições coloridas de triângulos com métodos computacionais e algébricos (Capítulo \ref{cap:triangles}).
Neste contexto, muitos dos problemas (ver Questões \ref{q:repeated-C4} e \ref{q:repeated-T}) se concentram em resultados assintóticos sobre o número mínimo de cores $f_k(n,H)$ para evitar $k$ repetições de $H$ no grafo $K_n$, enquanto nós exibimos soluções exatas de $f_2(n,K_3)$ que dependiam dos parâmetros.


Sobre o segundo problema investigado, enfatizamos que este trabalho fez uso importante dos resultados de Nenadov,
Person, Škorić e Steger~\cite{NePeSkSt14}. 
Auxiliados pelo trabalho deles, o problema de estimar o limiar da propriedade anti-Ramsey do bipartido completo no grafo aleatório se tornou em uma análise determinística de grafos com densidades limitadas. 
Acreditamos que o uso deste procedimento possa ser replicado para mais classes de grafos. 
No entanto, há uma dificuldade em transpor as informações algébricas de que a densidade do grafo é limitada para informações estruturais do grafo que permitam um método de coloração engenhoso.

Pretendemos continuar estudando as Conjecturas \ref{conj:antibipartidos} e \ref{conj:antiregulares} visando generalizar os resultados conhecidos, porém mais ferramentas parecem ser necessárias para tal feito. 
Este cenário de pesquisa da propriedade anti-Ramsey possui bastante potencial de descobertas, considerando que poucas classes de grafos puderam ser resolvidas.
Por outro lado, existe também a possibilidade de estudarmos o limiar para esta propriedade anti-Ramsey no grafo aleatório com ferramentas probabilísticas já bem difundidas como a regularidade, este tratamento já tem sido feito em outros trabalhos, vide \cite{KoKoMo12,araujo2022anti, aigner2022large}.

\appendix
\chapter{Casos Pequenos}\label{cap:app}

Neste Apêndice apresentamos os conjuntos $\Slr$ 
que são soluções escritas como pares não-ordenados de inteiros, $\{a,b\}$, para a desigualdade~(\ref{eq:pairs-ab})
da Proposição~\ref{lemma:inter_a} cobrindo os casos que denominamos como pequenos, i.e., quando $3 \leq \ell < 7$. 
Tal informação, em vista dos conjuntos apresentados na prova do Lema~\ref{lema:config}, completa o cenário de grafos bipartidos completos com mais de seis vértices.
%
Além disso, se $\ell \ll r$, as intersecções possíveis seguem os tipos A, B e C que encontramos na prova do Lema~\ref{lema:config} e as provas sobre suas respectivas configurações são análogas.  

Primeiro, exibimos os conjuntos $\mathcal{S}_{3,r}$:
\begin{equation*}%[label=\rmlabel]
     \begin{aligned}
        \mathcal{S}_{3, r}
            &:=
    \begin{cases}
        \mathcal{S}_{3,3} \cup \{\{2,4\}, \{3,3\}\}, 
        & \text{se }  r=4   \\ 
        %
        \mathcal{S}_{3,4} \cup \{\{2,5\}, \{3,4\}\}, 
        & \text{se }  r=5    \\ 
        %
        \mathcal{S}_{3,5} \cup \{\{2,6\}, \{3,5\}\}, 
        & \text{se }  r=6    \\ 
        %
         \mathcal{S}_{3,6} \cup \{\{3,6\}\}, 
        & \text{se }  r=7    \\ 
        %%
		\mathcal{S}_{3,5} \cup \{\{3,\beta\}: 4 \leq \beta \leq r-1\}, & \text{se }  8\leq r \leq 10 \\
		%%
        \mathcal{S}_{3,4} \cup \{\{3,\beta\}: 4 \leq \beta \leq r-1\}, & \text{se }  r \geq 11
        \end{cases}
    \end{aligned}  
\end{equation*}

A partir deste ponto usamos os seguintes conjuntos auxiliares, $\beta_1$ e $\beta_2$ seguem a formulação dada em (\ref{eq:conjBeta}):
\begin{equation*}
    \begin{aligned}
        I_1(\ell,r) &:= \{\{a,b\}: a=\ell\text{
    e }b\in \beta_1\}
        \\
        I_2(\ell,r) &:= \{\{a,b\}: a=\ell\text{ e }b\in \beta_1;\text{ ou }a=\ell-1\text{
    e }b\in \beta_2\}.
    \end{aligned}
\end{equation*}

Os conjuntos $\mathcal{S}_{4, r}$ seguem a formulação:
\begin{equation*}%[label=\rmlabel]
    \begin{aligned}
        \mathcal{S}_{4, r}
            &:=
  \begin{cases}
  \mathcal{S}_{4,4} \cup \{\{3,5\}, \{4,4\}\}, 
        & \text{se }  r=5      \\ 
        %
         \mathcal{S}_{4,5} \cup \{\{3,6\}, \{4,5\}\}, 
        & \text{se }  r=6      \\ 
        %%
         \mathcal{S} \cup \{\{3,\alpha\}: 3\leq\alpha\leq7\}\cup \{\{4,\beta\}: 4\leq \beta\leq r-1\}, 
        & \text{se }  r\in\{7,8\}      \\ 
        %
        %
        \mathcal{S}_{4,7} \setminus\{\{3,7\}\}, 
        & \text{se }  r=9      \\ 
        %
        %
         \mathcal{S}_{4,9} \setminus\{\{3,6\}\}, 
        & \text{se }  r=10      \\ 
        %
        %
         \mathcal{S}_{4,9} \setminus\{\{3,5\}\}, 
        & \text{se }  r=11      \\ 
        %
        %
         \mathcal{S} \cup \{\{3,3\}\} \cup \{\{4,\beta\}: 4\leq \beta\leq r-1\}, 
        & \text{se }  12\leq r\leq 15     \\ 
        %
        %
         \mathcal{S}_{3,3} \cup I_1(\ell,r),  & \text{se }  r \geq 16  
	\end{cases}
    \end{aligned}  
\end{equation*}

Por último, a descrição de $\mathcal{S}_{\ell, r}$ para $\ell \in \{5,6\}$:
\begin{equation*}%[label=\rmlabel]
    \begin{aligned}
        \mathcal{S}_{\ell, r}
            &:=
    \begin{cases}
        \mathcal{S}_{5,5} \cup \{\{4,6\}, \{5,5\}\}, 
        & \text{se } \ell=5 \text{ e } r=6     \\
        %%
         \mathcal{S}_{3,3} \cup \{\{3,3\}\}\cup I_2(\ell,r), 
        & \text{se } \ell=5 \text{ e } r=7    \\
        %%
         \mathcal{S}_{3,3} \cup I_2(\ell,r), 
        & \text{se } \ell=5 \text{ e } 8\leq r\leq 10    \\
        %%
         \mathcal{S}_{3,3} \cup I_1(\ell,r), 
        & \text{se } \ell=5 \text{ e } r\in \{11,12\}    \\
        %%
         \mathcal{S}_{3,3} \cup I_2(\ell,r),  & \text{se } \ell=6 \text{ e } r \in\{7,8\}  \\
         %%
        \mathcal{S} \cup I_2(\ell,r),  & \text{se } \ell=6 \text{ e }  9\leq r \leq 12 \\
         %%
        \mathcal{S} \cup I_1(\ell,r),  & \text{se } \ell \in\{5,6\} \text{ e } r \geq 13         
        \end{cases}
    \end{aligned}  
\end{equation*}



% ---------------------------------------------------------------------------- %
% Bibliografia
\backmatter \singlespacing   % espaçamento simples
\bibliographystyle{amsplain}% citação bibliográfica alpha
\bibliography{bibliografia}  % associado ao arquivo: 'bibliografia.bib'

\end{document}
